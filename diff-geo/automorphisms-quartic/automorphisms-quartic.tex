\input{/Users/daniel/github/config/preamble.sty}%This is available at github.com/danimalabares/config
\input{/Users/daniel/github/config/thms-eng.sty}%This is available at github.com/danimalabares/config

\begin{document}

\begin{minipage}{\textwidth}
	\begin{minipage}{1\textwidth}
		Differential Geometry Seminar \hfill IMPA
		
		{\small\hfill\href{https://github.com/danimalabares/seminars}{github.com/danimalabares/seminars}}

		
		%{\small\hfill\href{https://github.com/Friday-seminar/}{github.com/Friday-seminar}}
	\end{minipage}
\end{minipage}\vspace{.2cm}\hrule

\vspace{10pt}
{\Huge Two talks by Carolina Araujo}

\tableofcontents

\clearpage
\addcontentsline{toc}{section}{Automorphisms of quartic surfaces and Cremona transformations}
{\Huge Automorphisms of quartic surfaces and Cremona transformations}

\hfill{\Large Carolina Araujo}

{\Large \hfill IMPA}

\hfill{\large September 24, 2024}


\subsection{Motivation}

$X$ a smooth hypersurface of degree $d $, $X\subset \mathbb{P}^{n+1}$. We want to understand the group $\operatorname{Aut}(X)$. These are invertible polynomial maps from $X$ to itself. $X$ is defined by a single polynomial equation of degree $d$.

\begin{thm}[Matsumura-Mousky, 1964]\leavevmode
	Except in two special cases, all automorphisms of $X$ come from automorphisms of the ambient space:
	
	If $(n,d)\neq (1,3),(2,4)$, there is a surjective map
	\[\operatorname{Aut}(\mathbb{P}^{n+1},X)\overset{\pi}{\longrightarrow}\operatorname{Aut}(X)\]
	where $\operatorname{Aut}(X,Y)$ means automorphisms of $X$ that stabilize  $Y$.
\end{thm}

\subsection{Exceptional cases}

Let's look at the exceptionals cases.

\begin{enumerate}
	\item $(n,d)=(1,3)$. In this case  $C=X_3\subset \mathbb{P}^2$ is an elliptic curve and we have
		\[\operatorname{Aut}(C)\cong C\rtimes \mathbb{Z}_m\quad m=2,4,6\]
and
 \[\operatorname{Aut}(\mathbb{P}^2,C)\text{ is finite.} \]

 Elements in $C$ are translations of the torus. The \textit{\textbf{translation of $x$ with respect to $p$ and $(x:y:z)$}} is done by intersecting the curve with the line that joins $p$ and $(x:y:z)$ and reflecting. So we have a map
  \[t_p(x:y:z)=(F_1(x,y,z),F_2(x,y,z),F_3(x,y,z)\]
  We have created a \textit{\textbf{Cremona transformation}} (=biholomorphic birational map?).
 
  \begin{defn}[Cremona group]
	\[\operatorname{Bir}(\mathbb{P}^n)=\{\varphi :\mathbb{P}^n\overset{\operatorname{bir}}{\longrightarrow}\mathbb{P}^n:\text{bimeromorphic map} \}\]
\end{defn}

And we then have the surjective map
\[\operatorname{Bir}(\mathbb{P}^2,C)\overset{\pi}{\longrightarrow}\operatorname{Aut}(C)\]


	\item $(n,d)=(2,4)$. Here  $S=X_4=\mathbb{P}^3$ is a smooth quartic surface.

		\paragraph{Problem} (Gizatullin) Which automorphisms of $S$ are induced (not necesarily by automorphisms of $\mathbb{P}^3$) but at least by Cremona transformations of $\mathbb{P}^3$? ie. are restrictions of $\varphi\in\operatorname{Bir}(\mathbb{P}^3,S)$

\begin{remark}
	Related to K3 surface structure,
	\[\operatorname{Bir}(\mathbb{P}^3,S)\overset{\pi}{\longrightarrow}\operatorname{Bir}(S)\cong \operatorname{Aut}(S)\]
\end{remark}

$S$ is a K3 surface. $H^{2}(S,\mathbb{Z})\cong \mathbb{Z}^{22}$ and the Picard group acts on this lattice.

\begin{defn}[Picard rank of $S$]
	\[\rho(S)=\operatorname{rk}(\operatorname{Pic}(S))\in\{1,\ldots,20\}\]
\end{defn}

\begin{itemize}
\item If $S$ is very general, then $\rho(S)=1$ and $\operatorname{Aut}(S)=\{1\}$.

\item We are interested in $\rho(S)\geq 2$.
\end{itemize}
\end{enumerate}

\begin{example}[Oguiso, 2012]\leavevmode 
	\begin{enumerate}
		\item $\rho(S)=2$. Any cremona transformation that stabilizes the quadric is the identity:
			\[\operatorname{Aut}(S)=\mathbb{Z}\qquad \operatorname{Bir}(\mathbb{P}^3,S)=\{1\}\]
	
			
		\item $\rho(S)=3$,
			\[\operatorname{Aut}(S)=\mathbb{Z}_2*\mathbb{Z}_2*\mathbb{Z}_2\]
			All automorphisms are induced by Cremona transformations, ie. there is a surjective map
			\[\operatorname{Bir}(\mathbb{P}^3,S)\overset{\pi}{\longrightarrow}\operatorname{Aut}(S)\]

		\item (Paiva, Quedo 2023) Constructed surfaces with $\rho(S)=2$, $\operatorname{Aut}(S)=\mathbb{Z}_2$ and $\operatorname{Bir}(\mathbb{P}^3,S)=\{1\}$.
	\end{enumerate}
\end{example}

\begin{thm}[A-Paiva-(Socrates) Zika]\leavevmode
	Solution of Giztallin S problem for $\rho(S)=2$.
\end{thm}

\begin{remark}
	Not exactly, but "there is a moduli space for K3 surfaces of dimension $20-\rho(S)$". The generic case is $\rho(S)=1$.
\end{remark}

\subsection{K3 surfaces}

\begin{defn}
	A \textit{\textbf{K3 surface}} is a smooth projective surface that is simply connected and has a nowhere vanishing symplectic form $\omega\in H^{0}(S,\Omega^2_S)$
\end{defn}

\subsubsection{Lattices of $S$}

A \textit{\textbf{lattice}} is a finitely-generated abelian group with a nondegenerate pairin.

\begin{enumerate}
	\item $H^{2}(X,\mathbb{Z})\cong \mathbb{Z} \overset{\pi}{\hookleftarrow}\operatorname{Pic}(S) $. And if we tensor this with $\mathbb{C}$ we get $H^{2}(C,\mathbb{C})$, which admits a Hodge decomposition.

		Let's study automorphisms of a K3 surface. Let $g\in\operatorname{Aut}(S)$. It yields an element $g^*$ that acts on cohomology, ie $g^*\in\mathcal{O}(H^{2}(X,\mathbb{Z}))$ preserving the Hodge decomposition. This is called \textit{\textbf{Hodge isometry}}.

\begin{thm}[Global Torelli theorem]\leavevmode
	\begin{itemize}
	\item The correspondence $g\mapsto g^*$ is injective 
	
	\item If $\varphi\in\mathcal{O}(H^{2}(X,\mathbb{Z}))$ is a Hodge isometry preserving the ample classes, then $\varphi=g^*$ for some $g\in\operatorname{Aut}(S)$.
	\end{itemize}
\end{thm}
\end{enumerate}

\begin{example}
	$S\subset \mathbb{P}^3$ smooth quartic surface with $\rho(S)=2$. Using the equivalence of line bundles module isomorphism and curves modulo intersection,
	\[\operatorname{Pic}(S)= \left<H,C\right> \]
	where $H$ is a hyperplane section of $\mathbb{P}^3$. Then
	\[Q=\begin{pmatrix} H^2& H\cdot C\\H\cdot C& C^2 \end{pmatrix} =\begin{pmatrix} 4&b\\b&2c \end{pmatrix} \]
	using that $2g(c)-1$, so the number in the lower-right on RHS is even.
\end{example}

\begin{remark}
	Hodge Index Them $Q$ is rank (1,1).
\end{remark}

\begin{defn}[Discriminant]
	\[r=\operatorname{disc}(S)=-\det Q\]
\end{defn}

\begin{prop}
	$S$ general K3 surface (not needed that it is a quartic) with $\rho(S)=2$. Then
	\[\operatorname{Aut}(S)=\begin{cases}
		\{1\}\qquad &\text{(finite)}  \\
		\mathbb{Z}_2\qquad &\text{(finite, Dani-Ana)} \\
		\mathbb{Z}\qquad&\text{(infinite, Oguiso 1)}  \\
		\mathbb{Z}_2*\mathbb{Z}_2\qquad &\text{(infinite, Dani-Ana)} 
	\end{cases}\]
	where the first two are characterized by containing a en elliptic curve or a rational? curve, ir. $\exists D\in\operatorname{Pic}(S)$ such that $D^2=0,-2$. On the other hand, the last two cases are distinguised by $\not\exists D\in\operatorname{Pic}(S)$ such that $D^2=0,-2$.

	\[\exists \sigma\in\operatorname{Aut}(S)\text{ of order 2}\iff \exists  \text{ ample }A\in\operatorname{Pic}(S) \text{ such that }A^2=2.\]
\end{prop}

\begin{remark}
	In low Picard numbers there are no symplectic involutions…?
\end{remark}

So to understand the surface we want to understand those bundles and that is all in the discriminant (not in the lattice itself!).

\begin{remark}
	$\exists D\in\operatorname{Pic}(S)$ s.t. $D^2=k \iff x^2-ry^2=4k$  has integer solutions.
\end{remark}

Given the quadratic form we can find the automorphisms.

\subsubsection{Main theorem}

\begin{thm}[A-Paiva-Zika]\leavevmode
	$S\subset \mathbb{P}^3$ general smooth quartic surface with $\rho(S)=2$ and $\operatorname{disc}(S)=r$.
\begin{enumerate}
	\item (Negative answer to Gizatulla's problem) If $r>57$ or  $r=52$ then
		 \[\operatorname{Bir}(\mathbb{P}^3,S)=\{1\}\]
		 So we cannot realize any automorphism as a Cremona transformation.

	\item If $r\leq 57$ and $r \neq 52$, then we get the full automorphism group, ie. a surjective map
		\[\operatorname{Bir}(\mathbb{P}^3,S)\overset{\pi}{\longrightarrow}\operatorname{Aut}(S)\]
\end{enumerate}
\end{thm}

\subsection{Birational geometry}

Take the case of
\[\begin{tikzcd}
	\operatorname{Bir}(\mathbb{P}^2)=\left<\operatorname{Aut}(\mathbb{P}^2),q\right> \\
	\operatorname{Aut}(\mathbb{P}^2)\arrow[u,hook]
\end{tikzcd}\]
and the map
\begin{align*}
	q: \mathbb{P}^2 &\overset{\operatorname{bir}}{\longrightarrow}\mathbb{P}^2  \\
	(x:y:z) &\longmapsto \left(\frac{1}{x}:\frac{1}{y}:\frac{1}{z}\right)=(yz:xz:xy)
\end{align*}
which is well-known (Noether-Castelnuovo). So that is a decomposition of automorphisms and quadratics. Now in greater dimension, $n\geq 3$ we have

\begin{thm}[Sakisov Program]\leavevmode
	(The theorem is much more general) Any $\varphi \in\operatorname{Pic}(\mathbb{P}^n)$ can be factorized with  \textit{\textbf{Sarkisov links}}  $\varphi_i$:
	\[\begin{tikzcd}
		\mathbb{P}^n\arrow[r,dashed,"\varphi_1"]\arrow[rrrr,bend left,"\varphi"]&X_1\arrow[r,dashed,"\varphi_2"]\arrow[d]&X_2\arrow[r,dashed]\arrow[d]&\cdots \arrow[r,dashed]&X_k=\mathbb{P}^n\\
		&T_1&T_2
	\end{tikzcd}\]
\end{thm}

Now look at $n=3$, $\operatorname{Bir}(\mathbb{P}^3,S)\subset \operatorname{Bir}(\mathbb{P}^3)$. This is a Calabi-Yau pair:

\begin{defn}[Calabi-Yau pair]\leavevmode 
	A pair $(X,D)$
	\begin{itemize}
	\item Terminal projective variety.
	\item $K_X+D\sim0$ that is, a meromorphic top form that does not vanish on hypersurface and has simple pole on $D$. Then $(X,D)$ is called \textit{\textbf{log canonical}}.
	\end{itemize}

Now take two Calabi-Yau pairs $(X,D_X)$ and  $(Y,D_Y)$.

 \begin{align*}
	\operatorname{div}(\omega_{D_X})&=-D_X\\
	\operatorname{div}(\omega_{D_Y})&=-D_Y
\end{align*}
We say that a birrational map $f:X\to Y$ is \textit{\textbf{volume preserving}} is  $f_*\omega_{D_X}=\omega_{D_Y}$.

\end{defn}


\begin{thm}[Volume-Preserving]\leavevmode
	Everything like in the Sarkisov theorem but now maps are volume-preserving.
\end{thm}

In our case, $(\mathbb{P}^3,S)$, we can classify the v.p. Sakisov links from  $(\mathbb{P}^3,S)$. It starts by blowing up a curve $C\subset S$. But this curve has genus and degree very restricted, it's something like
\[(g(C),\operatorname{deg}(C))\in\{(0,1),(0,2),\ldots,(11,10),(14,11)\}\]

So for the main theorem, it was checked that if $r>57$ there are no curves from the list. And in the second item of the main theorem, there exist these curves, for instance curve  $(14,11)$ for rank 56, then produce a link that starts by blowing it up and magically gives you the Cremona transformation that restricts with automorphism with which you started.

\begin{remark}
	So perhaps we expect the answer to G. problem to be almost never.
\end{remark}

\begin{question}
	How does that blowing-up work?
\end{question}

\[\begin{tikzcd}
	X\arrow[r,"\text{flops}" ]\arrow[d,"\operatorname{Bl}_C"]& X\arrow[d,"\text{contract?}" ]\\
	C\subset S\subset \mathbb{P}^3\arrow[r,dashed]&\mathbb{P}^3\supset S'\supset C'
\end{tikzcd}\]
So for example in case $(2,8)$ you obtain something of the same type.

\clearpage
\addcontentsline{toc}{section}{Birational geometry of Calabi-Yau pairs}
{\Huge Birational geometry of Calabi-Yau pairs}

\hfill{\Large Carolina Araujo}

{\Large \hfill IMPA}

\hfill{\large 31 January, 2025}

\subsection{Pairs \((X,D)\)}
Itaca's program (1970's): \(U\) complex algebraic variety. We want to find invariants like Kodaira dimension. Compactify \(U \rightsquigarrow X\), and consider \(X\setminus U:=D\) (the boundary). The \(\Omega^q_X(\operatorname{log}D)\)

\begin{thm}[Itaca, 1977]\leavevmode
Kodaira dimension of the pair \((X,D)\) does not depend on the choice of compactification (and it exists ;)
\end{thm}

\begin{defn}[Lu-Zhang, 2017]\leavevmode
\((X,D)\) is \textit{\textbf{Brody-hyperbolic}} \textit{\textbf{(Mori-hyperbolic)}} if there is no nonconstant holomorphic (analytic) morphism  \(f:\mathbb{C} \to X\setminus D\) and the same holds for the open strata of \(D\).
\end{defn}

\begin{conjecture}
\((X,D)\) is Brody-hyperbolic then \(K_X + D\) is ample.
\end{conjecture}

\begin{thm}[Sraldi, 2019]\leavevmode
\((X,D)\) Mori-hyperbolic then \(K_X+D\) is nef.
\end{thm}

\begin{question}\leavevmode
How to make sense of ``the birational geometry of \((X,D)\)"
\end{question}

\subsection{Calabi-Yau pairs}

\begin{defn}[Calabi-Yau pair]\leavevmode
We relax a condition: we won't ask that \(X\) is smooth, but that it is a terminal projective variety. And also that \(K_X+D \sim 0\) (I think this ``would be equivalent to" Kodaira dimension 0). Also ask that \((X,D)\) is log canonica.
\end{defn}

\begin{remark}\leavevmode
The condition \(K_X+D \sim 0\) says that there is unique up to scaling volume form \(\omega_D \in \Omega^m_{\mathbb{C}(X)}\) that does not vanish and has a simple pole along \(D\), i.e.  \(\operatorname{div}(\omega_D)=-D\).
\end{remark}

Now we study the birational geometry of CY pairs:

\begin{defn}\leavevmode
Let \((X,D_X),(Y,D_Y)\) CY pairs and \(f: X \dashrightarrow Y\) a birational map. We get a pullback \(f_*:\Omega^n_{\mathbb{C}(X)}\overset{\cong}{\to} \Omega^n_{\mathbb{C}(Y)}\). Then \(f:(X,D_X) \dashrightarrow (Y,D_Y)\) is \textit{\textbf{volume preserving}} is \(f_*(\omega_{D_X}=\lambda \omega_{D_Y}\).
\end{defn}

\begin{remark}[Valuative characterization]\leavevmode

\end{remark}

\begin{defn}\leavevmode
\((X,D)\) CY pair,
\[\operatorname{Bir}(X,D):=\{f \in \operatorname{Bir}(X) : f:(X,D) \dashrightarrow (X,D) \text{ is volume preserving} \}\]
\end{defn}

\subsection{The Sarkisov-Program}

\(\operatorname{Bir}(\mathbb{P}^n)\) Cremona group.

\begin{thm}[Noether-Castelnuovo, 1870-1901]\leavevmode
The creoma group of \(\mathbb{P}^2\) has a nice set of generators: maps of degree 1 and \(q\):
\[\operatorname{Bir}(\mathbb{P}^2) = \left<\operatorname{Aut}(\mathbb{P}^2),q\right>\]
where \(q\) is the \textit{\textbf{standard quadratic transformation}} given by \((x:y:z) \overset{q}{\mapsto }(yz:xz:xy)\)
\end{thm}

You might think that because it has such a nice group generators it'd be easy to study this group. It's not the case.

Now in dimension 3:

\begin{thm}[Hilda Hudson (1927)]\leavevmode
\(\operatorname{Bir}(\mathbb{P}^n)\) does not admit a set of generators of bounded degree.
\end{thm}

\begin{thm}[Sarkisov Program, Corti 1995; Hazon-McKunam 2013 for \(n \geq 4\)]\leavevmode
	\[\begin{tikzcd}
		\mathbb{P}^n\arrow[r,dashed,"\varphi_1"]\arrow[d]\arrow[rrrr,bend left,"\varphi"]&X_1\arrow[r,dashed,"\varphi_2"]\arrow[d]&X_2\arrow[r,dashed]\arrow[d]&\cdots \arrow[r,dashed]&X_k=\mathbb{P}^n\arrow[d]\\
	\{\operatorname{pt}\}	&T_1&T_2&&\{\operatorname{pt}\}
	\end{tikzcd}\]
The intermediate varieties \(X_i/T_i\) are called \textit{\textbf{Mori-fiber spaces}}, and \(\varphi_i\) are \textit{\textbf{Sarkisov links}}. So the theorem is that a map (what map?) can be factorized by Sarkisov links.
\end{thm}

\begin{thm}[Corti-Kalog(?) 2016]\leavevmode
Any volume-preserving birrational map between Mori-fiber CY pairs is a composition of volume-preserving Sarkisov links. Now every intermidiate variety admits a divisor making it a CY pair:
	\[\begin{tikzcd}
		(X,D_X)\arrow[r,dashed,"\varphi_1"]\arrow[d]\arrow[rrrr,bend left,"\varphi"]&(X_1,D_1)\arrow[r,dashed,"\varphi_2"]\arrow[d]&(X_2,D_2)\arrow[r,dashed]\arrow[d]&\cdots \arrow[r,dashed]&(Y,D_Y)\arrow[d]\\
	T_X	&T_1&T_2&&T_Y
	\end{tikzcd}\]
\end{thm}

\begin{remark}\leavevmode
\begin{itemize}
\item The Sarkisov links
\[\begin{tikzcd}
	 \mathbb{P}^n\arrow[r, dashed]\arrow[d]&X_1\arrow[d]\\\operatorname{pt}&T_1
\end{tikzcd}\]
are not classified.

\item Depending on \(D\), the volume-preserving Sarkisov links
\[\begin{tikzcd}
	 \mathbb{P}^n\arrow[r, dashed]\arrow[d]&X_1\arrow[d]\\\operatorname{pt}&T_1
\end{tikzcd}\]
can be classified.
\end{itemize}
\end{remark}

\begin{thm}[A-Coti-Massareti]\leavevmode
\(D \subset \mathbb{P}^n\) a \textit{general} hypersurface of degree \(n+1\). If you want to study the birational group of the pair \((\mathbb{P}^n,D)\), \(\operatorname{Bir}(\mathbb{P}^n,D)\), then this group is not interesting. Because \(\operatorname{Bir}(\mathbb{P}^n,D)=\operatorname{Aut}(\mathbb{P}^n,D)\).

And then also consider \(D\) smooth instead of general, and with Picard rank \(\rho(D)=1\).
\end{thm}

\begin{thm}[A-C-M]\leavevmode
\(D \subset \mathbb{P}^3\) general (A1 singularity and \(\rho(S)=1\)) quartic with a singularity. \(\operatorname{Bir}(\mathbb{P}^3,D)\cong \mathbb{G} \rtimes \mathbb{Z}/2\mathbb{Z}\), where \(\mathbb{G}\) is ``form of \(\mathbb{G}m\) over \(\mathbb{C}(x,y)\)".
\end{thm}

\subsection{Last application}

\(X=X_d \subset \mathbb{P}^{n+1}\) smooth hypersurface of degree \(d\). I want to study the automorphism group.

\begin{thm}[Matsumata-Monsky, 1964]\leavevmode
Except for the two cases when \(\mathcal{H}\), \((n,d)=(1,3)\) or \((2,4)\), any smooth automorphism is the restricition of an automorphism of the ambient space, i.e. \(\operatorname{Aut}(\mathbb{P}^{n+1},X)\twoheadrightarrow \operatorname{Aut}(X)\).

In the exceptional cases: for \((n,d)=(1,3)\) we get \(\operatorname{Bir}(\mathbb{P}^2,C) \twoheadrightarrow \operatorname{Aut}(C)\) {\color{6}\(C\) is a curve}. For \((n,d)=(2,4)\), we  \textit{ask}: when \(\operatorname{Bir}(\mathbb{P}^3,S) \twoheadrightarrow \operatorname{Aut}(S)\)?
\end{thm}

\begin{thm}[A-Paiva-Zika]\leavevmode
Complete solution \(\rho(S)=2\).
\end{thm}










\end{document}
