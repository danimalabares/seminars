\input{/Users/daniel/github/config/preamble.sty}%available at github.com/danimalabares/config
\input{/Users/daniel/github/config/thms-por.sty}%available at github.com/danimalabares/config
\renewcommand*{\contentsname}{}  

\begin{document}

\begin{minipage}{\textwidth}
	\begin{minipage}{1\textwidth}
		Ergodic Theory Seminar \hfill IMPA
		
		{\small\hfill\href{https://github.com/danimalabares/seminars}{github.com/danimalabares/seminars}}
	\end{minipage}
\end{minipage}\vspace{.2cm}\hrule

\vspace{10pt}

{\Huge Ação de homeos sobre o grafo fino de curvas: uma classificação}

\vspace{1em}
\hfill{\Large Pierre-Antoine Guihéneuf}

\hfill{\large Sorbonne Université}

\hfill{\large 24 de Outubro de 2024}

\vspace{-2em}
\tableofcontents

\section{Grafo fino de curvas }

Fixe $S$ uma superfície fechada de género $g \geq 1$. O \textit{\textbf{grafo fino de curvas}} $C^+(S)$ (tem um grafo que não é fino que é exatamente a mesma coisa só que \textit{up to homotopy}) é:
\begin{itemize}
\item Vértices: curvas simples fechadas não homotopicamente triviais em $S$.
\item Arestas: $\alpha-\beta$ sse $\#(\alpha\cap \beta)\leq 1$.
\end{itemize}
Tem uma distância natural em $C^+(S)$.

\begin{remark}\leavevmode
$\operatorname{Homeo}(S)\curvearrowright C^+(S)$ dada por $h\alpha=h(\alpha)$ é uma ação por isometrias.
\end{remark}

\begin{remark}\leavevmode
$C^+(S)$ não é localmente compacto.
\end{remark}

\begin{thm}[BHW21 $\ddot\smile$]\leavevmode
	$C^+(S)$ é conexo, de diametro infinito e Gromov-hiperbólico.
\end{thm}

\begin{defn}\leavevmode
	Um grafo é \textit{\textbf{Gromov-hiperbólico}} se $\exists \delta>0$ tal que todos os triangulos geodésicos são $\delta$-finos. 
\end{defn}

\begin{defn}\leavevmode
	Triangulo $\delta$-fino: que pode pegar uma vizinhança de raio $\delta $ de qualquer par de lados, e o lado restante fica conteúdo na união daquelas vizinhanças.
\end{defn}

\section{Um pouco sobre espaços Gromov-hiperbólicos}

\begin{example}\leavevmode
	\begin{itemize}
	\item Arvores
	\item $\mathbb{H}^2$
	\end{itemize}
\end{example}

\begin{thing3}{Classificação das isometrias}\leavevmode
	\begin{itemize}
		\item \textit{\textbf{Eliptica}} se existe uma órbita de diametro finito, i.e. $\exists x  \in X$ tal que $\operatorname{di a m}\{f^n(x)\}<+\infty$.
	\item \textit{\textbf{Loxodrómica}} se $\exists x \in X$ tal que 
		\[\lim_{n \to \infty} \frac{d(f^n(x),x)}{n}\]
	\item \textit{\textbf{Parabólica}} o resto.
	\end{itemize}
\end{thing3}


\begin{defn}\leavevmode
	Uma \textit{\textbf{quase-isometria}} de $X$ é $g:X\to X$ tal que $\exists \lambda,c$ 
	\[\lambda^{-1}d(x,y)-c\leq d(g(x),g(y))\leq \lambda d(x,y)+c\]
\end{defn}

\begin{thing7}{O bordo de $X$}\leavevmode
	
\end{thing7}

\begin{defn}\leavevmode
	O \textit{\textbf{bordo de $X$}} é
\begin{align*}\partial X&=\{\text{ mergulhos $Q$ de $R_+$ em $X$} \} \Big/\text{ dist. Hausdorff finita}\\
	&=\{\text{conjunto de direções no infinito} \}
	\end{align*}
\end{defn}

\begin{prop}\leavevmode
	$X\cup  \partial X$ é completo.
\end{prop}

\begin{thing6}{Pergunta}\leavevmode
	Forma de $\partial C^+(S)$?
\end{thing6}

\begin{remark}\leavevmode
	Toda isometria de $X$ se extende em um homeo de $X\cup \partial X$.
\end{remark}

\begin{thm}[Gromov]\leavevmode
	\begin{itemize}
	\item Se $f$ é parabólica então $f$ tem um único ponto fixo em $X\cup \partial X$, qué esta em $\partial X$.

	\item Se $f$ é loxodrómico, então $f$ tem dois pontos fixos em $X \cup \partial X$ que estão no bordo e a dinámica é norte-sul. $f\Big|_{\partial X\setminus \{a^-\}}$ é uma contração.
	\end{itemize}
\end{thm}

\begin{remark}[Dani]\leavevmode
	Parece que tem uma dinâmica dada simplesmente pela iteração da $f$. Supongo que essa $f$ é um isomorfismo.
\end{remark}

$f^{-1}|_{\partial X\setminus \{a^+\}}$ é uma contração. Isso permete jogar ping-pong.

\begin{exercise}\leavevmode
	$\exists n$ tal que $\left<f^n,g^n\right> $ é livre.
\end{exercise}

\begin{thm}[Hensel, le Raux (private communication)]\leavevmode
	Se $f,g\in\operatorname{Homeo}(\Pi^2)$, $\rho(f),\rho(g)$ com interior não vazio, $\rho(f),\rho(g)$ não são os mesmos ao menos de homotecia/traslação. Então existe $n$ tal que  $\left<f^n,g^n\right> $ é livre e $\forall h\in\left<f^n,g^n\right> $, $\operatorname{i n t}(\rho(h))\neq \varnothing$.
\end{thm}

\section{Classificação}

\begin{thm}[]\leavevmode
$f\in\operatorname{Homeo}_0(S)$. Então aspse
\begin{enumerate}[label=(\roman*)]
	\item $f$ age sobre $C^+(S)$ de maneira loxodromica.
	\item $\exists P\subset S$ finito, $f$-invariante tal que $f|_{S\setminus P}$ é pseudo-Anosov.
	\item $\operatorname{i n t}\rho_{\operatorname{erg}}(f)\neq \varnothing$.
\end{enumerate}
\end{thm}

\begin{thm}\leavevmode
	$f\in\operatorname{Ho m eo}_0(\Pi^2)$. Então $f$ age de maneira elíptica see $f$ tem desvios limitados em uma direção racional $v\in\mathbb{Z}^2\setminus\{0\}$, $\exists C,p\in\mathbb{R}^{2}$ tais que $\forall n,x$,
	\[|\left<\tilde{f}^n(x)-n\rho,v\right> |\leq C\]
\end{thm}
\end{document}
