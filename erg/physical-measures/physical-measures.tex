\input{/Users/daniel/github/config/preamble.sty}%available at github.com/danimalabares/config
\input{/Users/daniel/github/config/thms-eng.sty}%available at github.com/danimalabares/config

\begin{document}

\begin{minipage}{\textwidth}
	\begin{minipage}{1\textwidth}
		Ergodic Theory Seminar \hfill IMPA
		
		{\small\hfill\href{https://github.com/danimalabares/seminars}{github.com/danimalabares/seminars}}

		
		%{\small\hfill\href{https://github.com/Friday-seminar/}{github.com/Friday-seminar}}
	\end{minipage}
\end{minipage}\vspace{.2cm}\hrule

\vspace{10pt}

{\Huge Substitution dynamics on infinite alphabets}

\hfill{\Large Ali Messãoudi}

{\Large \hfill UNESP}

\hfill{\large November 28, 2024}

\begin{thing4}{Abstract}\leavevmode
A substitution is a map from an alphabet A to the set
of finite words in A. To any substitution we can naturally associate
a symbolic dynamical system that is well studied in the literature
when the alphabet is a finite set and connected to several areas such
as ergodic theory and number theory among others. In this work,
we study ergodic and geometric properties of dynamical systems
associated to substitutions in infinite alphabets. This study involves
finite and infinite invariant measures, countably infinite matrices
and Rauzy Fractals.	
\end{thing4}

\tableofcontents

\section{Physical measures}

$X$ compact measure space, $m$ a reference measure, $f:X \to X$, $x \in X$.
\[e_n(x):=\frac{1}{n}\sum_{n=0}^{n-1}\delta_{f^i(x)}\]
é a \textit{\textbf{sequência canônica de empirical measures}}.

Se $e_n(x)$ converge a $\mu$. Então $\mu$ descreve a \textit{estadística}  de $x$. O \textit{\textbf{basin}} de $\mu$ $\mathcal{B}_\mu$ é o conjunto de pontos $x \in X$ tal que $e_n(x)\to \mu$. $\mu$ é uma \textit{\textbf{physical measure}} se $m(\mathcal{B}_{\mu})>0$.

\begin{question}\leavevmode
	O que acontece se $e_n(x) $ não converge?
\end{question}
Devem existir duas medidas $\mu,\nu$ e $n_i,n_j\to\infty$ tais que $e_{n_i}(x)\to \mu$ e $e_{n_j}\to \nu$. Dizemos que $x$ tem um comportamento \textit{\textbf{não estadístico}}. Se $m$  é tal que quase todo ponto é não estadístico, $m$ é \textit{\textbf{não estadística}}.

\section{Exemplos}

Bota muitos zeros, depois muito mais uns, depois muuuito mais zeros…



\end{document}
