\input{/Users/daniel/github/config/preamble.sty}%available at github.com/danimalabares/config
\input{/Users/daniel/github/config/thms-eng.sty}%available at github.com/danimalabares/config

\begin{document}

\begin{minipage}{\textwidth}
	\begin{minipage}{1\textwidth}
		Ergodic Theory Seminar \hfill IMPA
		
		{\small\hfill\href{https://github.com/danimalabares/seminars}{github.com/danimalabares/seminars}}

		
		%{\small\hfill\href{https://github.com/Friday-seminar/}{github.com/Friday-seminar}}
	\end{minipage}
\end{minipage}\vspace{.2cm}\hrule

\vspace{10pt}

{\Huge Substitution dynamics on infinite alphabets}

\hfill{\Large Ali Messãoudi}

{\Large \hfill UNESP}

\hfill{\large November 28, 2024}

\begin{thing4}{Abstract}\leavevmode
A substitution is a map from an alphabet A to the set
of finite words in A. To any substitution we can naturally associate
a symbolic dynamical system that is well studied in the literature
when the alphabet is a finite set and connected to several areas such
as ergodic theory and number theory among others. In this work,
we study ergodic and geometric properties of dynamical systems
associated to substitutions in infinite alphabets. This study involves
finite and infinite invariant measures, countably infinite matrices
and Rauzy Fractals.	
\end{thing4}

\tableofcontents

\section{Introdução para quem não é da área}

\begin{defn}\leavevmode
	Um \textit{\textbf{sistema dinâmico}}  é um par $(X,f)$ com  $X \neq  \varnothing$ and $f:X\to X$.
\end{defn}

Em sistemas dinâmicos estamos interessados em estudar a órbita
\[O(x)=\{f^n(x), n \in \mathbb{N}, f^n:=f \circ f \circ\ldots \circ f\}\]
para $x \in X$.

Quando tem uma medida invariante, i.e. $\mu(f^{-1}(A)=\mu(A),; \forall A \in \mathcal{B}$,  quase todo ponto volta perto dele mesmo.

\section{Introdução de verdade}

Considere um alfabeto $A$ que é só um conjunto finito. $A^*$ são as palavras finitas. $A^\mathbb{N}$ são as palavras infinitas. $A^\mathbb{Z}$ são as palavras infinitas pra diante e pra atrais.

Uma \textit{\textbf{substitução}} é um mapa
\begin{align*}
	\sigma: A &\longrightarrow A^* \\
	a &\longmapsto \sigma(a)
\end{align*}

\begin{lemma}\leavevmode
	Existe $u \in A^\mathbb{N}$ e $k \in \mathbb{N}^*$ tal que $\sigma^k(u)=u$ é um ponto periódico de $\sigma$.
\end{lemma}

\begin{remark}\leavevmode
	A substituição de Fibonacci $A=\{0,1\}$ $\sigma(0)=01, \sigma(1)=0$ tem um ponto fixo, o limite de $\sigma(0)$, ie.
	\[u:=\lim_{n \to \infty} \sigma^n(0),\qquad \sigma(u)=u.\]
	Pode codificar a informação da substituição $\sigma$ numa matriz: cada renglón é o numero de vezes que aparece cada letra do alfabeto na imagem de $\sigma$.
\end{remark}

\section{Classical results}

