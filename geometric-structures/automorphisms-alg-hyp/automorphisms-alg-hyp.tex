\input{/Users/daniel/github/config/preamble.sty}%available at github.com/danimalabares/config
\input{/Users/daniel/github/config/thms-eng.sty}%available at github.com/danimalabares/config

\begin{document}

\begin{minipage}{\textwidth}
	\begin{minipage}{1\textwidth}
		Geometric Structures on Manifolds Seminar\hfill IMPA
		
		{\small\hfill\href{https://github.com/danimalabares/seminars}{github.com/danimalabares/seminars}}
		
		%{\small\hfill\href{https://github.com/Friday-seminar/}{github.com/Friday-seminar}}
	\end{minipage}
\end{minipage}\vspace{.2cm}\hrule

\vspace{10pt}

{\Huge Automorphisms of algbraically hyperbolic manifolds}

\hfill{\Large Misha Verbitsky}

\hfill{\large November 21, 2024}

\tableofcontents
\begin{upshot}\leavevmode
using dynamic argument, hyperkähler is non hyperbolic. In the end it is beacuse of automorphism group. We shall see how to prove it is finite.
\end{upshot}
\begin{thing3}{Past upshot}\leavevmode
	
Recall that last week we talked about Kobashi hyperbolicity, which means that Kobayashi pseudo-distance is non-degenerate. We saw that this implies Brody hyperbolicity, which means that $X$ has no non-constant holomorphic maps. If $X$ is compact this is an equivalence.

Today we shall see that this implies algebraic hyperbolicity. Demially's conjecture is that this last implication is actually iff.
\end{thing3}

\begin{thing3}{Abstract}\leavevmode
	Kobayashi hyperbolic manifold is a compact Hermitian manifold $M$
 such that any holomorphic map from the Poincare disk to $M$
 is C-Lipschitz, with a fixed constant C. Such manifolds admit a metric, called ``Kobayashi metric'', which is invariant under all holomorphic automorphisms; this immediately implies that the group of holomorphic automorphisms is compact. It is not hard to see that compactness implies that this group is finite. Algebraic hyperbolicity is a seemingly weaker algebraic version of this notion which was first defined by J.-P. Demailly. Conjecturally, it is equivalent to Kobayashi hyperbolicity. I will explain why algebraically hyperbolic manifolds have finite fundamental group. This is a result obtained jointly with F. Bogomolov and L. Kamenova. I will explain how it implies that hyperkahler manifolds with Picard rank $\geq 3$
 are not algebraically hyperbolic.
\end{thing3}

\section{Introduction}


\begin{question}\leavevmode
	What is up with Picard rank $\geq  3$ hypothesis?
\end{question}

\begin{thing4}{Another characterizarion of Koba hyp}\leavevmode
	$(M,I,\omega)$ compact, Hermitian manifold such that there exists $C>0$ and for each holomorphic map from $(\Delta,$Poincaré metric $)$ to $M$ is $C$-Lipschitz.
\end{thing4}
\begin{proof}\leavevmode
Because we bound kobayashi pseudo distance from below with the hermitian/riemannian metric, making it into a distance. I think I read this sometime ago in some Kobayashi book.
\end{proof}

\begin{claim}\leavevmode
	If kobayashi sectional curvature is bounded from below by $-\varepsilon$ with $\varepsilon\gg 0$ then any map $\Delta \to M$ is $C$-Lipschitz with $C=\frac{c}{\varepsilon}$
\end{claim}

\begin{thm}[Brody]\leavevmode
if $M$ is compact Kobayahi hyperbolic is equivalent to not having a copy of $\mathbb{C}$.
\end{thm}

\section{Review/other perspective of algebraically hyperbolic}

\begin{defn}\leavevmode
	$M$ projective manifold. (makes 0 sense for Kähler). $M$ is called \textit{\textbf{kobayashi hyperbolic}} if for each compact curve  $S \subset M$, 
	\[\int_{S}M\omega\leq (g-q)C\]
	for some $C>0$
\end{defn}

\begin{remark}\leavevmode
	So Lucas just put degree, but the integral is the intersection number with hyperplane sections. (the two definitions \textit{are} equivalent)
\end{remark}

\begin{question}\leavevmode
	 H o o o ow!? To pass from the integral to the intersection form.
\end{question}

\begin{thm}[Who?]\leavevmode
Kobayashi hyperbolic implies algebraically hyperbolic.
\end{thm}

\begin{thing4}{conjecture}[Demailly]\leavevmode
	converse implication holds. He proposed a scheme to prove this that was later proved wrong by his students.
\end{thing4}

now we prove

\begin{thm}[Who?]\leavevmode
Kobayashi hyperbolic implies algebraically hyperbolic.
\end{thm}

\begin{proof}\leavevmode
\begin{enumerate}[label=\textbf{Step \arabic*}]
\item Start with a curve $C$ of genus $>1$. Then $C=\Delta/\Gamma$. Then $\operatorname{Vol} C=-\int_{C}c_1(c)=(g-1)\alpha$. Where $\alpha$ is probably $2\pi$---just a constant. \textit{Because curvature is Chern class! $\ddot\cup $}. So that proves it.

\item Suppose $M$ is Kobayashi hyperbolic compact Hermitian manifold. "By compactness" there exists a constant $\varepsilon$ such that $h \leq \varepsilon h_K$, where  $h$ is normal hermitian metric and $h_K$ is Kobayashi metric.

 \item Let $j:S \hookrightarrow M$ be a curve in a projective manifold, its genus is $\geq 2$  by hyperbolicity. Notice $j$ is 1-Lipschitz with respect to Kobayashi hyperbolic metric because it is holomorphic. Now
	 \[(g-1)\alpha=\operatorname{Vol}_{K}(S)\geq \operatorname{Vol}_{KM}(S)\geq \varepsilon\operatorname{Vol}_{\text{Fubini-Study}}S \]
This gives $g-1\geq \frac{\varepsilon}{2 \alpha}\operatorname{deg} S$ proving hyperbolicity. 
\end{enumerate}
\end{proof}

\section{The automorphism group of an algebraically hyperbolic manifold is discrete}

\begin{claim}\leavevmode
	Any hyperbolic manifold $M$ has discrete automorphism group.
\end{claim}

\begin{proof}\leavevmode
\begin{enumerate}[label=\textbf{Step \arabic*}]
\item The group of automorphisms of a projective manifold is a complex Lie group. Its connected component of identity $G_0$ {\color{4}Couldn't follow}
\end{enumerate}
\end{proof}

\begin{thm}[Not in slides]\leavevmode
$X$ projective admits a self map $f:X \to X$ of degree $\operatorname{deg}f>1$ then $X$ it not algebraically hyperbolic.
\end{thm}
\begin{proof}\leavevmode
Degree is the number o preimages (or the image of the fundamental class in cohomology) so we have $\operatorname{deg} S \operatorname{deg} f$ and "the same for genus"
\end{proof}

\begin{claim}\leavevmode
	The automorphism group of a compact Kobayashi hyperbolic manifold is finite.
\end{claim}

\begin{proof}\leavevmode
Compactness and using the claim whose prof I didn't follow.
\end{proof}

\begin{thm}[with F. Bogomolov and L. Kamenova]\leavevmode
The automorphism group of an algebraically hyperbolic manifold is finite.
\end{thm}

Proof is in three statements.

\begin{prop}\leavevmode
	If there is an automorphism of $M$ not preserving the rational Kähler class then $M$ is not algebraically hyperbolic. (If the image of $\operatorname{Aut}(M)$ in $\mathsf{GL}(H^{1,1}(M,\mathbb{R}))$ does not preserve any rational Kähler class, then $M$ is not algebraically hyperbolic.)
\end{prop}

\begin{thing4}{Proposition 2}[The trickyiest]\leavevmode
	The image of the automorphism group in $\mathsf{GL}(H^{1,1}(M))$ is finite and image of automorphism group in $\operatorname{Aut}(\operatorname{Pic}(M))$ infinite. Then $M$ is noa algebraically hyperbolic.
\end{thing4}

\begin{thing3}{Proposition 3}\leavevmode
	$\operatorname{Aut}(M)$ infinite, image in $\operatorname{Aut}(\operatorname{Pic}(M))$ finite. Then $M$ is not algebraically hyperbolic.
\end{thing3}

\begin{upshot}\leavevmode
	Torus dynamics breaks hyperbolicity.
\end{upshot}

\section{HyperKähler case}

\begin{thm}[Them?]\leavevmode
$M$ hyperkähler with $\operatorname{rk}(\operatorname{Pic}(M))\geq 3$ then $M$ is not algebraically hyperbolic.
\end{thm}

\begin{proof}\leavevmode
Quadraitc lattice has infinietly many automorphisms (infinite automorphism group) if $(1,n)$ fr  $n\geq  2$. And in thie case we have that the picard  lattice is such a lattice; on board:
\[\operatorname{Sym}(M)\cong\mathsf{O}(H^{1,1}(M,\mathbb{Z}),q)\]
where $\operatorname{Sym}(M)$ is holomorphic symplectic automorphisms.

\end{proof}

































\end{document}
