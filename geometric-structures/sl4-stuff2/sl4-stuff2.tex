\input{/Users/daniel/github/config/preamble.sty}%available at github.com/danimalabares/config
\input{/Users/daniel/github/config/thms-eng.sty}%available at github.com/danimalabares/config


\begin{document}

\begin{minipage}{\textwidth}
	\begin{minipage}{1\textwidth}
		Geometric Structures on Manifolds \hfill IMPA
		
		{\small\hfill\href{https://github.com/danimalabares/seminars}{github.com/danimalabares/seminars}}

		
		%{\small\hfill\href{https://github.com/Friday-seminar/}{github.com/Friday-seminar}}
	\end{minipage}
\end{minipage}\vspace{.2cm}\hrule

\vspace{10pt}

{\Huge  $\operatorname{SL}(4)$-stuff (Part 2)}

{\Large Investigations of $\operatorname{SL}(4)$-structures}


\hfill{\Large Graham Smith}

{\Large \hfill PUC-Rio}

\hfill{\large October 10, 2024}

\vspace{1em}
We continue our investigation into the structure of convex cocompact representations of compact surface subgroups in semi-simple Lie groups, focusing this time on the anti de Sitter Case. We shall show how Thurston's constructions, including measured geodesic laminations and real trees, are realised geometrically inside anti de Sitter space, and we will see how k-surfaces serve to interpolate between these structures. Time permitting, we will study how these structures generalize to (1,3)-Anosov representations in PSL(4,R).

{\color{6}\bfseries Abstract.}\hspace{.5em} 

\tableofcontents

\section{Anti de Sitter space}

\[\operatorname{A d S}^{3,1}=\{(x_1,x_2,x_3,x_4):x_1^2+x_2^2-x_3^2-x_4^2=-1\}\]
Projectivize everything and take an affine chart $(y_1,y_2,y_3)$. you compute that $y_1^2+y_2^2-y_3^2=1-1/x_4^2$. This is the inside of a hyperboloid in $\mathbb{R}^{3}$. Remember that it is missing one point because it's an affine chart. So in reality it's a solid torus.

\clearpage

{\Huge  $\operatorname{SL}(4)$-stuff (Part 2)}

{\Large On complete maximal submanifolds in pseudo-hyperbolic space} 


\hfill{\Large Graham Smith}

{\Large \hfill PUC-Rio}

\hfill{\large February 4, 2025}

\subsection{Teichmüler theory}

Nowadays we think of Teichmüler space as
\[\{ \rho: \Gamma\to \mathsf{PSL}(2,\mathbb{R}): \phi \text{ discrete, injective} \} \Big/ \mathsf{PSL}(2,\mathbb{R})\]

Consider
\[\mathsf{PSL}(2,\mathbb{R})\to \text{a reductive Lie group} \]
(just think of \(\mathsf{SL}(n)\) or \(\mathsf{U}(n)\))

First consider
\[\{\phi:\Gamma \to G:\rho \text{ injective, discrete, Anosov/convex cocompact} \}\Big/G\]
for a surface group \(\Gamma=\pi_{1}(\Sigma)\)

\subsection{Generalizing domain \(\Gamma\)}

So instead of surface group consider \(\Gamma\) a Gromov hyperbolic group, so \(\Gamma=\pi_{1}(X^d)\) for some \(X\) compact hyperbolic.

\begin{enumerate}
\item Quasi-Fuschian manifolds, \(\Gamma : = \pi_{1}(X^d)\), \(\rho: \Gamma \to \mathsf{PSO}(d+1,1)\). For \(d=2\), Ahlfors-Bers.

\item More recently, Barbot 2015. \(\Gamma= \pi_{1}(X^d)\), \(\rho: \Gamma \to \mathsf{PSO}(d,2)\) e.c. Connected component of Fuschian repns consist of convex co-compact repns.

	Mostow: deformation space of \(\pi_{1}(X^d) = \Gamma \subseteq \underbrace{\mathsf{PSO}(d,1)}_{\text{isometries of \(\mathbb{H}^{d}\)} }\) 


	\begin{thing4}{Summary so far}\leavevmode
	All deformations of quasiFuschian representations are convex cocompact( that means that a cocompact action on a convex thing).
	\end{thing4}
\end{enumerate}

\subsection{Beyrer-Kassel '23}

\(\Gamma\) gromov hyp. \(\partial  \Gamma=\mathbb{S}^{p-1}\). Consider the representation space
\[\{\rho:\Gamma\to \mathsf{PSO}(p,q+q)\}\Big/\mathsf{PSO}(p,q+1)\]
Then I think that the representations that are convex cocompact are:
\[\{\text{ convex cocompact} \}=\text{union of (I think \textit{some}) conneced components.} \]
\subsection{Towards our result}

\begin{quotation}
	to understand a group you study the space it acts on
\end{quotation}

So take \(\mathsf{PSO}(p,q+1)\) and the space it acts on is
\[\mathbb{H}^{p,q}:=\{(x,y) \in \mathbb{R}^p \times \mathbb{R}^{q+1}: \|x\|^2+\|y\|^2=1\}\]
\[=\{x \in \mathbb{R}^{p,q+1}: \|x\|^2_{p,q+1}=-1\}\]
Now have a look at different cases
\begin{enumerate}
\item \(q=0\) \(\mathbb{H}^{p,0}=\mathbb{H}^{p}\).
\item \(q=1\) \(\mathbb{H}^{p,1}=\operatorname{AdS}^{p,1}\)
\end{enumerate}

\subsubsection{Anti-De sitter space}

\begin{quotation}
	In a coordinate chart, it looks like the inside space of a one sheeted hyperbola.
\end{quotation}

\begin{itemize}
\item Maximal spacelike totalic geodesic slices are copies of \(\mathbb{H}^{p}\) 
\item Maximal timelike totally geodesic slices are copies of \(\mathbb{S}^q\) with \(-1_X\) metric.
\end{itemize}

Now we look at \textbf{Fermi coordinates}: a foliation of \(X\) by (I think) geodesics orthogonal to \(Y\). So if  \(Y\) is a point then it's just geodesic coordinates.

We have taken Fermi coordinates about totally geodesic timelike sphere. It's an exercise to compute that the metric is
\[g=g^p_{\operatorname{hyp}}-\operatorname{co s h}^2(R)g_{\operatorname{ s p h}}^q \]

Then we look at stereographic coordinates…


Eventually we have discussed the space
\[\mathcal{M}= \{ M^p =\mathbb{H}^{p,q}: \text{\(M\) spacelike, maximal (minimal?), complete} \}\]
and the \textit{\textbf{ideal boundary map}} 
\[\partial_\infty: \mathcal{M}\to B\]
which is continuous and
\begin{thm}\leavevmode
\(\partial_\infty\) is a homeomorphism,

\(\forall  B \in \mathcal{\Big\}\exists ! M \in \mathcal{M}\) \text{ such that \(\partial _\infty M= B\) (so a plateau i.e. given the boundary put the manifold inside with some curvature conditions.} 
\end{thm}






















\end{document}
