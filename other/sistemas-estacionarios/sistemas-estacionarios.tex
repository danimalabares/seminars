\input{/Users/daniel/github/config/preamble-por.sty}%available at github.com/danimalabares/config
\renewcommand*{\contentsname}{}  

\begin{document}

\begin{minipage}{\textwidth}
	\begin{minipage}{1\textwidth}
		Ergodic Theory Seminar \hfill IMPA
		
		{\small\hfill\href{https://github.com/danimalabares/seminars}{github.com/danimalabares/seminars}}
	\end{minipage}
\end{minipage}\vspace{.2cm}\hrule

\vspace{10pt}

{\Huge Medidas estacionárias de um Sistema Dinâmico Aleatório no círculo}

\vspace{1em}
\hfill{\Large Graccyela Rosybell Salcedo Pirela}

\hfill{\large ICMC}

\hfill{\large 26 de Septembro de 2024}

\vspace{-2em}
\tableofcontents

\section{Exemplos}

Rotações  com pendente racional ou irracional. Sistema com pontos atratores e repulsores.

\section{Conf}
	$\mathcal{F}\subset \operatorname{Hom}(\mathbb{S}^1)$, $ \nu$ probabilidade, $\operatorname{Hom}(\mathbb{S}^1)$, $\operatorname{supp}(\nu)=\mathcal{F}$. Espaço de probabilidade:
	\[(\Omega,\mathbb{P})=(\mathcal{F}^{\mathbb{N}},\nu^{\mathbb{N}})\]
	Caminho aleatório:
	\[\Omega \ni \omega\longmapsto(f^n_\omega)_{n\geq 0}\]
	onde $\omega=(f_n)_{ n \in \mathbb{N}}$, $f_{\omega}^n:=f_n\circ \ldots \circ f_1$,  $f^0_\omega =\operatorname{id}_{\mathbb{S}^1}$.

\section{Ponto de vista topológico}

 \[S=\{f^n_{\omega}:\omega \in\Omega,n \geq 0\}\]
 $A\subset \mathbb{S}$ é \textit{\textbf{invariante}} se  $f(A)\subset A$, $\forall f\in S$. Um conjunto fechado $A\subset \mathbb{S}^1$ é chamado \textit{\textbf{minimal}} invariante se  $\forall B\subset A$ tal que $B$ é invariante então $B=A$ ou  $B=\varnothing $.
  \[O(x)=\{f(x):f\in S\}\]

 \section{Ponto de vista probabilístico}

$(f^n_\omega(x))$. $\eta\in\operatorname{Prob}(\mathbb{S}^1)$, dizemos que é  \textit{\textbf{$ \nu$-estacionária}} se
\[\eta=\int f_*\eta d\nu(f).\]
\[T:(\omega,x)\longmapsto(\sigma\omega,f'_\omega(x))\]
$\eta$ é $\nu$-estacionária  $\iff$ $\mathbb{P}\otimes \eta$ e $T$-invariante. Dizemos que $\eta$ é \textit{\textbf{$\nu$-ergódica e estacionaria}} se $\mathbb{P}\otimes \eta$ é $T$-ergódica.

\section{Resultados previos}

\begin{idea1}{Resultado previo}[Furst, 63]\leavevmode
	$\forall x:\mathbb{P}$ $\omega$
	\[d\left( \frac{1}{N}\sum_{n=0}^{N-1} \delta_{p^n_\omega x} \right) \subset \nu \text{-estacionária}  \]
\end{idea1}

\begin{idea1}{Resultado previo}[Malicet '17]\leavevmode
	Tricotomía:
	\begin{enumerate}
	\item $\not \exists m\in\operatorname{Prob}(\mathbb{S}^1)$, $f_*m=m\;\forall f\in\mathcal{F}\implies $ contração local $\implies $ $\exists d\in\mathbb{N}$, $\operatorname{Prob}_{\nu}^{\operatorname{erg}}(\mathbb{S}^1)=\{\eta_1,\ldots,\eta_d\}$ 
			\[k:=\operatorname{supp}(\eta)\]
			\[\implies U_i(x)=\mathbb{P}\left(\omega\frac{1}{N}\sum_{n=0}^{N-1}\delta_{p_\omega^\eta}\right)\overset{\omega}{\longrightarrow}\eta_1\]
	
		\item O semigrupo $\delta$ é topológico conjugado como um semigrupo das isometrías atuando minimalmente ($\mathbb{S}^1$ é mínima).

			\[\exists ! \text{ medida $\nu$-esacionária} \]

		\item $\exists x$ $O(x)=\{f(x),\ldots,f\in\mathbb{S}\}$ infinita.
	\end{enumerate}
\end{idea1}

\begin{idea4}{Exemplos}\leavevmode
	\begin{enumerate}
		\item $f_1=R_{p/q}$.
		\item Círculo con atrator no polo norte e repulsor no polo sul.
		\item [Desenho] Parece que são círculos com dois pontos mais o fluxo sai de cada ponto a um lado dele e entra do outro lado.
	\end{enumerate}
\end{idea4} 

\section{$|O(x)|= \infty\;\forall x$}

\[m^* _x=\lim_{N\to \infty}\frac{1}{N}\sum_{n=0}^{N-1}\int \delta_{\rho^\eta_\omega x}d\mathbb{P}(\omega)=\sum_{i=1}^d U_i(x)\eta_i\]
$x\longrightarrow m_x$ contínua.

\begin{enumerate}[label=\textbf{$d=$\arabic*}]
	\item $x\longrightarrow \eta_1$.
	\item $x\longrightarrow m_x$ 
	\item[$d>2$] [Desnhos de triangulo e quadrado]
\end{enumerate}

\begin{idea1}{Teorema 1}\leavevmode
	$\forall x\; |O(x)|= \infty$ $\forall i=1,\ldots,d$ $ \exists \mathcal{A}_{i}$ família finita de intervalos fechados tal que
	\begin{enumerate}
		\item $K_1\subset \{U=1\} =\bigcup_{I\in A_i}I $
	
		\item $i\neq j$, $I\cap J=\varnothing,\;\forall J\in A_i,J\in A_j$.
		\item $\forall a,b,$ $a\in J\in A_i,$ $b \in J\in A_j$ tal que $(a,b)\cap \bigcup_{i=1} ^d\bigcup_{J\in A_i} I\neq \varnothing$.
			\begin{enumerate}[label=(\roman*)]
				\item $U_k=0$ em $[a,b]$  $\forall k\neq 1,0$.
				\item $U_i,U_j$ são monótonos sobre $(a,b)$. 
			\end{enumerate}
	\end{enumerate}
\end{idea1}

\begin{coro}
	$\forall x$ $|O(x)| =\infty$, $\{\eta_x:x\in\mathbb{S}^1\} \subset \{t\eta_1+(1-t)\eta_j:N\in\{1,\ldots,d\}, t\in [0,1]\}$.
\end{coro}

\begin{idea7}{Proposição 2}\leavevmode
	$I\subset \mathbb{S}^1$ intervalo, $i\in\{1,\ldots,d\}$. Se $U_1|_{I}\equiv C_f$
\end{idea7}

\begin{proof}[Do teorema 1]
	
\end{proof}

\end{document}
