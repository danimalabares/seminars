\input{/Users/daniel/github/config/preamble.sty}%available at github.com/danimalabares/config
\input{/Users/daniel/github/config/thms-eng.sty}%available at github.com/danimalabares/config

\begin{document}

\begin{minipage}{\textwidth}
	\begin{minipage}{1\textwidth}
		Ergodic Theory Seminar \hfill IMPA
		
		{\small\hfill\href{https://github.com/danimalabares/seminars}{github.com/danimalabares/seminars}}

		
		%{\small\hfill\href{https://github.com/Friday-seminar/}{github.com/Friday-seminar}}
	\end{minipage}
\end{minipage}\vspace{.2cm}\hrule

\vspace{10pt}

{\Huge Existence and non-existence of physical measures for doubly intermittent maps}


\hfill{\Large Stefano Luzzato
}

{\Large \hfill ICTP}

\hfill{\large November 28, 2024}

\begin{thing4}{Abstract}\leavevmode
We introduce a large class of one-dimensional map in
the topological conjugacy class of 2x mod1 but exhibiting a variety
of ergodic behaviours, such as the existence of invariant probability
measures equivalent to Lebesgue, Dirac-delta physical measures and,
perhaps most interestingly, non-existence of physical measure. This
class generalises the standard well known and well studied PomeauManneville and Liverani-Saussol0-Vaienti maps.
\end{thing4}

\tableofcontents

\section{Physical measures}

$X$ compact measure space, $m$ a reference measure, $f:X \to X$, $x \in X$.
\[e_n(x):=\frac{1}{n}\sum_{n=0}^{n-1}\delta_{f^i(x)}\]
é a \textit{\textbf{sequência canônica de empirical measures}}.

Se $e_n(x)$ converge a $\mu$. Então $\mu$ descreve a \textit{estadística}  de $x$. O \textit{\textbf{basin}} de $\mu$ $\mathcal{B}_\mu$ é o conjunto de pontos $x \in X$ tal que $e_n(x)\to \mu$. $\mu$ é uma \textit{\textbf{physical measure}} se $m(\mathcal{B}_{\mu})>0$.

\begin{question}\leavevmode
	O que acontece se $e_n(x) $ não converge?
\end{question}
Devem existir duas medidas $\mu,\nu$ e $n_i,n_j\to\infty$ tais que $e_{n_i}(x)\to \mu$ e $e_{n_j}\to \nu$. Dizemos que $x$ tem um comportamento \textit{\textbf{não estadístico}}. Se $m$  é tal que quase todo ponto é não estadístico, $m$ é \textit{\textbf{não estadística}}.

\section{Exemplos}

Bota muitos zeros, depois muito mais uns, depois muuuito mais zeros…



\end{document}
