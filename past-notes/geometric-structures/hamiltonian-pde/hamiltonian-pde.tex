\input{/Users/daniel/github/config/preamble.sty}%available at github.com/danimalabares/config
\input{/Users/daniel/github/config/thms-eng.sty}%available at github.com/danimalabares/config

\begin{document}

\begin{minipage}{\textwidth}
	\begin{minipage}{1\textwidth}
		Geometric Structures on Manifolds Seminar\hfill IMPA
		
		{\small\hfill\href{https://github.com/danimalabares/seminars}{github.com/danimalabares/seminars}}
		
		%{\small\hfill\href{https://github.com/Friday-seminar/}{github.com/Friday-seminar}}
	\end{minipage}
\end{minipage}\vspace{.2cm}\hrule
\vspace{10pt}

{\Huge Some problems in Hamiltonian geometry of PDEs}

\hfill{\Large Raffaele Vitolo}

\hfill{\large December 5, 2024}

\begin{thing4}{Abstract}\leavevmode
The Hamiltonian formulation of Partial Differential Equations is one of the cornerstones of the theory of Integrable Systems. Being carried out by analogy with finite-dimensional Hamiltonian systems, it has an intrinsic geometric nature. In this talk we will review differential-geometric aspects of the Hamiltonian theory of PDEs as well as new projective-geometric properties of known Integrable Systems that are emerging in recent years.
\end{thing4}

\tableofcontents

\section{Introduction to the Hamiltonian formalism of PDEs}

Integrablity of PDEs is defined as the existence, for a given equation, of an infinite sequence of symmetries or conserved quantities in involution.

\begin{itemize}
\item KDV equation. Gardner, Greene, Kruskal, Muira. \textit{Method for solving the KdV Equation}. 
\end{itemize}

An evolutionary system of PDEs
 \[F=u^i_t-f^i(t,x,u^j,u^j_x,u^j_{xx},\ldots)=0\]
 {\color{4}(so no variables $u$ depending on $t$ inside the parenthesis)} admits a Hamiltonian formulation if there exist $A$, $\mathcal{H}=\int h dx$ such that
 \[u^i_t=A^{ij}\left( \frac{\delta \mathcal{H}}{\delta u^j} \right)\]   where $ \frac{\delta \mathcal{H}}{\delta u^j}=(-1)^\sigma\partial_\sigma\frac{\partial h}{\partial u^j_\sigma}$ is the variational derivative. Here $A=(A^{ij})$ is the \textit{\textbf{Hamiltonian operator}} which is like the Poisson tensor: {\color{4}an classical finite-dimensional mechanics, $A$ is the Poisson tensor, i.e. inverse of the matrix of the tangent-cotangent isomorphism induced by a symplectic form.}
 
\section{Hamiltonian operators}
$A$ is a Hamiltonian operator iff
\[\{F,G\}_A=\int \frac{\delta F}{\delta u^i}A^{ij\sigma} \partial_{\sigma}\frac{\delta G}{\delta u^j}dx\]
is a Poisson bracket (skew-symmetric and Jacobi).

\begin{remark}\leavevmode
	There is an intrinsic definition of differential operators given by Grothendieck. If you want a map of modules $\Delta:A\to B$ to be a differential operator, take an element of their common underlying ring $\rho$ and the multiply-by-$\rho$ operator $\delta_\rho:A \to B$ and ask that
	\[[\delta_{\rho_1},[\delta_{\rho_2},\ldots [\delta_{\rho_k},\Delta]\ldots]=0.\]
	
\end{remark}

Anyway, there's two conditions that characterize the bracket being Poisson: skew-adjoint $A^*=-A$ and another named after someone.


\begin{example}[KdV equation] \leavevmode
	Take
	\[u_t=u u_x+u_{x x x}\]
There are two Hamiltonian structures:
\[H_1=\frac{u^3}{6}+\frac{u^2_x}{2},\qquad H_2=\frac{u^2}{2}\]
with 
 \[A_1=\partial_x,\qquad  A_2=\frac{1}{3}u_x	+\frac{2}{3}u\partial_x+\partial_{x x x}\]
one first order operator, and the sum of one first order plus one third order. This is the model for several other cases of bi-Hamiltonian systems that we'll se in a moment.
\end{example}


\section{Motivation for Hamiltonian PDEs}
A Hamiltonian operator maps \textit{conservation laws}  to \textit{symmetries}.

Magri: two  \textit{compatible } Hamiltonian operators $A_1,A_2$ generate a sequence of conserved quantities:
\[A_1\left( \frac{\delta H_{n+1}}{\delta u^i} \right) =A_2\left( \frac{\delta H_n}{\delta u^i} \right) \]
the operator maps the variational derivative into a symmetry.

\textbf{Integrability:} the above sequence $H_1,\ldots,H_n,\ldots$ is in involution:
\[\{H_i,H_j\}=0\]
which is the {\color{3}most important property}. What to do with this sequence? You can prove that you can integrate the equation once this sequence exists!

There is no analogue of Liouville theorem for PDEs, but maybe you can find good coordinates, or prove it is C-integrable via scattering transform.

\section{Structure of bi-Hamiltonian pairs}

Clsses of hamiltonian operators which are invariant under certain operators.

\section{First-order homogeneous operators}

These were introduced by Dubrovin and Novikov:
\[P^{ij}_1=g^{ij}(\underbrace{b^{ij}_k(\mathbf{u})u^k_x}_{\text{term with derivatives} }+\underbrace{b^{ij}_k(\mathbf{u})u^k_x}_{\text{free term} }.\]
(Tensor.) Every summand in the operator containd the same number of $x$ derivatives (homogeneous).

If we require --- we get ---:
\begin{itemize}
\item Skew-adjointness is equivalent to: symmetry of $g^{ij}$ and the connection $\Gamma$ is metric.
\item Jacobi identity holds iff connection $\Gamma$ is symmetric and pseudo-Riemannian metric is flat.
\end{itemize}

\begin{remark}\leavevmode
	A paper by Cartan (from 1932?) on classification of flat metric connections with torsion $T$.
\end{remark}

\section{Classification of bi-Hamiltonian pairs}
If we classify bi-Hamiltonian pairs we clasfiy hiechrachies of conservations laws.

Compatible triples (e.g. KdV), compatible paris (e.g. Monge-Ampère).

Compatibnle triple means

\[P_1=\partial_x,\qquad Q=?,\qquad   R=?\]

 \[A_1=P,\qquad  A_2=Q+R?\]


\section{Classification by projective reciprocal transformations}
 
We now put a group. We see that the operators of the higher order remain the same under the action of the group, while the lower-order ones change.

Second order and third order operators can be classified by means of this group. So fix an operator of higher order, then rest is classified.

Second order homogeneous Hamiltonian operators $R_2$ are in bijective correspondence with linear line congruences. Third order homogeneous Hamiltonian operators are in correspondence with ?

\textit{\textbf{Lie coordinates for Plücker embedding}} are given by: take two infinitesimally closed points, so their difference is a differential, then take minors of a matrix. Looks like you get  $\mathbb{P}\Lambda^{2}(V)$.

\section{Projective classification of triples}

Initiated by Lorenzoni, Savoldi, V. JPA 2017. Here we classify triples
\[A_1=P_1+R_2,\qquad  A_2=Q_1.\]
$R_2$ is second order, $P_1,Q_1$ are Ferapontov operators.

They are classified by a quadratic line complex. Probably the compatibility of the two varieties gives information about …

\begin{example}[Kaup-Broer system]\leavevmode
	The quadratic line complexes are conics. With this mechanism we can construct an integrable system.

	\begin{remark}[Dani]\leavevmode
		Looks like every operator is giving a metric.
	\end{remark}
\end{example}

Many fanous integrable systems, there are algebraic varieties that determine the integrable system.

	\begin{tabular}{c | c}
		Third order Hamiltonian operator& Quadratic line complex\\
Second order Hamiltonian operator & System of $n$ linear line complexes\\
$R_2$-comp. first order Hamiltionian operator& Quadratic line complex\\
$R_3$-comp. first-order Hamiltonian operator & ??
	\end{tabular}






\end{document}
