\input{/Users/daniel/github/config/preamble.sty}%available at github.com/danimalabares/config
\input{/Users/daniel/github/config/thms-eng.sty}%available at github.com/danimalabares/config


\begin{document}

\begin{minipage}{\textwidth}
	\begin{minipage}{1\textwidth}
		Geometric Structures on Manifolds \hfill IMPA
		
		{\small\hfill\href{https://github.com/danimalabares/seminars}{github.com/danimalabares/seminars}}

		
		%{\small\hfill\href{https://github.com/Friday-seminar/}{github.com/Friday-seminar}}
	\end{minipage}
\end{minipage}\vspace{.2cm}\hrule

\vspace{10pt}

{\Huge  $\operatorname{SL}(4)$-stuff (Part 2)}


{\Large Twistor space of a compact hypercomplex manifold is never Moishezon}


\hfill{\Large Yulia Gorjinyan}

{\Large \hfill IMPA}

\hfill{\large October 17, 2024}

\vspace{1em}

{\color{6}\bfseries Abstract.}\hspace{.5em}  Moishezon manifolds are compact, complex manifolds that admit many curves and divisors which can be used to study the geometry of the ambient manifold. Twistor spaces of compact hyperkahler manifolds are very far from being Moishezon. I am going to explain why the twistor space of a compact hypercomplex manifold is never Moishezon and neither Fujiki class C (in particular, never Kahler and projective). It is the work in progress.


\tableofcontents

\section{Twistor space}


Start with $M$ a riemannian oriented 4 manifold  and you have the \textit{\textbf{Hodge star operator}}:
\begin{align*}
	*: \Lambda^{2}(M) &\longrightarrow \Lambda^{2}(M) \\
	\alpha\wedge *\beta &g(\alpha,\beta)\operatorname{Vol}_M
\end{align*}
and it happens that
 \[*^2=1\implies \text{exist $\pm 1$ eigenspaces} \]

 \begin{remark}\leavevmode
 	\[\Lambda^{2}(M)\cong \mathsf{SO}(TM)\]
so $s\in\Lambda^{+}(M)$ is seen as an endomorphism of $TM$.
 \end{remark}

 Now consider the spherification
  \[Z:=S\Lambda^{+}(M)=\{\text{unit elements in }\Lambda^+ \}\]
Now put a complex structure 
\[T_{m,s}Z=T_mM\oplus T_sS\Lambda^{+}(M):=\mathfrak{I}_S\oplus \mathfrak{I}_{S\Lambda^{+}(M)}\]
using $s\in S\Lambda^{+}(M)\rightsquigarrow \mathfrak{I}_S$ using the former isomorphism. This makes
\[(Z,\mathfrak{I}_S\oplus \mathfrak{I}_{S\Lambda^{+}(M)}\]
an almost complex manifold. That is the \textit{\textbf{twistor space}}
\begin{question}\leavevmode
	When is it complex
\end{question}
Consider the Riemannian curvature tensor
\begin{align*}
	R: \Lambda^{2}(M) &\longrightarrow \Lambda^{2}(M) \\
	e_i\wedge e_j &\longmapsto \frac{1}{2} \sum_{k,\ell}R_{ijk\ell}e_k\wedge e_\ell
\end{align*}
\begin{thm}[Singer, Hopf?, 1969]\leavevmode
	The representation of the curvature tensor
	\[R\longmapsto(\operatorname{tr}(A,B,\underbrace{\frac{1}{2}A-\operatorname{tr}(A),}_{W^+},\underbrace{\frac{1}{3}C-\operatorname{tr}(C))}_{W^-}\]
	when $W^+$ is autodual component of the Riemannian curvature tensor.
then the twistor space complex. This is called an \textit{\textbf{ASD manifold} }

So $W^+=0$, $W=0$?
\end{thm}

\begin{thm}[N. Hitchin]\leavevmode
	The only twistor spaces which admits a Kähler metric are
	\begin{itemize}
	\item $Z=\mathbb{C}P^{3}$, $M=S^4$.
	\item $Z=\mathbb{F}(\mathbb{C}^3)$, $M=\mathbb{C}P^{2}$.
	\end{itemize}
\end{thm}

\section{Moishezon manifolds}

\begin{defn}\leavevmode
	A compact complex manifold is called \textit{\textbf{Moishezon}} if its birrationally equivalent to a projective manifold, i.e. there is  $\mu:\tilde{X}\to  X$ holomorphic birrational map.
\end{defn}

\begin{thm}[F. Campana]\leavevmode
	A twistor space is Moishezon only when the 4-manifold is $S^4$ or $\#_n\mathbb{C}P^{2}$.
\end{thm}

\section{Hypercomplex manifolds}

\begin{defn}\leavevmode
	A manifold $M$ is \textit{\textbf{hypercomplex}} if it has three integrable almost complex structures  $I$,  $J$, $K$ satisfying the quaternionic relations $I^2=J^2=K^2=-\operatorname{Id}$ and $I J=K=-J I$.
\end{defn}

From now on we assume $(M,I,J,K)$ is a compact hypercomplex manifold.

\begin{example}[Most interesting]\leavevmode
	A \textit{\textbf{Hopf manifold }} is
	\[\dfrac{\mathbb{C}^n\setminus \{0\}}{\left<\gamma\right> }\]
	where $\left<\gamma\right> $ is the cyclic group generated by holomorphic contractions. When $n$ is even and $\gamma\in\mathsf{GL}(\mathbb{H})$.
\end{example}

Now consider 
\[L=a I+b J + c K \]
with $a^2+b^2+c^2=1$ defines a $\mathbb{C}P^{1}$-family of complex structures called the \textit{\textbf{twistor deformations.}}

 \begin{defn}\leavevmode
	Let's $(M,I,J,K)$ be a hc manifold,
\[\operatorname{Tw}(M)\cong M\times \mathbb{C}P^{1}\]
\[M\times \mathbb{C}P^{1} \ni(x,L)\rightsquigarrow T_{(x,L)}M\times \mathbb{C}P^{1}\]
$L$ at  $T_xM$,  $\mathfrak{I}_{\mathbb{C}P^{1}}$ at $T_L\mathbb{C}P^{1}$.
\end{defn}


\begin{thm}[Salamon, aledin, Ibata]\leavevmode
	$(\operatorname{Tw}(M),L\oplus \mathfrak{I}_{\mathbb{C}P^{1}})$ is a complex manifold.
\end{thm}

\begin{thing5}{Examples}\leavevmode
	\begin{itemize}
		\item HKCR: $\operatorname{Tw}(\mathbb{H}^n)\cong \operatorname{Tot}\mathcal{O}(1)^{2n}\cong \mathbb{C}P^{2n+1}\setminus \mathbb{C}P^{2n-1}$.
		\item Compact complex manifold $X$, define a notion of algebraic dimension $a(X)$: it is the trascendental degree of the field of algebraic functions on $X$,  $k(X)$. W say  $X$ is Moishon if $a(X)=\dim_\mathbb{C}X$. This is a birrational invariant. So theorem by Moishon is that this and the other definition are equivalent. Which  becomes easy once you have Hironaka theorem. Also need Stein reduction.

			$\mathsf{OK}$ so take a Hopf surface, which is elliptic so it has a map to $\mathbb{C}P^{1}$ with elliptic fibers? $\mathsf{OK}$ so $a(X)=1$,  $a(X)=0$ (tot. hom. elliptic case).

			\begin{thm}[Pontecorro]\leavevmode
				Hopf surface is hypercomplex so has twistor space $X\rightsquigarrow \operatorname{Tw}(X)=Z$. Then $a(Z)=2$.
			\end{thm}
	\end{itemize}
\end{thing5}

Now let's compute the algebraic dimension of the general Hopf manifold.

\begin{defn}\leavevmode
	Let $X$ be a compact complex manifold. An \textit{\textbf{algebraic reduction}} $X^{\operatorname{red}}$ is a compact projective manifold $X^{\operatorname{red}}$ and a meromorphic dominant map (rational) $X \overset{\varphi}{\dashrightarrow}X^{\operatorname{red}}$ such that $\varphi^* :\operatorname{Mer}(X^{\operatorname{red}})=k(X^{\operatorname{red}}\overset{\cong }{\longrightarrow}\operatorname{Mer}(X)=k(X)$.

	\[\begin{tikzcd}
	&M\arrow[dl,"\text{bimerom} ",swap]\arrow[dr,"\text{proper} "]\\
		X\arrow[rr,"\varphi",dashed]&&X^{\operatorname{red}}
	\end{tikzcd}\]
\end{defn}

\begin{thm}[Verbitsky]\leavevmode
	The twistor space $\operatorname{Tw}(X)$ of a compact hyperkähler manfiold $M$ has an algabraic dimension $a(\operatorname{Tw}(M))=1$.
\end{thm}

\begin{question}\leavevmode
	If you take hypercomplex, can you get other algebraic dimensions? That is, what is possible algebraic dimension of twistor spaces?
\end{question}

\begin{thm}\leavevmode
	$X$ hypercomplex twistor cannot be Moishozon.
\end{thm}

\section{Algebraic dimension of the twistor space of a Hopf manifold}
{\color{5}This is an example we get $a(\operatorname{Tw}(X))=2n$ from an elliptic Hopf manifold $X^{2n}.$}

Let $X=\mathbb{H}^n\setminus \{0\} /\gamma$ be a hypercomplex Hopf manifold with is elliptic, meaning there is a map $X\to \mathbb{C}P^{2n-1}$ with elliptic fibers so
\begin{align*}
	:X  &\longrightarrow \mathbb{C}P^{2n-1} \\
	(z_1,\ldots,z_{2n} &\longmapsto [z_1,\ldots,z_{2n}]
\end{align*}
It's more less easy to see the the fibers are elliptic. Here holomorfic contraction acts as multiplication by diagonal matrix?  $\begin{pmatrix} \mu_1&& \\& \cdots & \\&&\mu_{n}\end{pmatrix} $.

\[\begin{tikzcd}
	&\operatorname{Tw}(\mathbb{H}^n\setminus \{0\} \cong \mathcal{O}(1)^{2n}\setminus \text{zero section} \arrow[dl,"\mu"]\arrow[dr,"\mathbb{C}^*\text{ action}" ]\\
	\operatorname{Tw}(\mathbb{H}^n\setminus \{0\}= \operatorname{Tw}(X)\arrow[dr,"\mathbb{C}^*\text{ action} "]&  &  \mathbb{P}(\mathcal{O}(1)^{2n})\arrow[dl,"\cong "]\\
	&\mathbb{C}P^{1}\times \mathbb{C}P^{2n-1}
\end{tikzcd}\]

Then $a(\operatorname{Tw}(X))\cong a(\mathbb{C}P^{1}\times \mathbb{C}P^{2n-1}$

\section{Hodge structures and polarization}

We need some variations of Hodge structures now.

Let $V_{\mathbb{Z}}$ be a free $\mathbb{Z}$-module and $V_{\mathbb{C}}=V_{\mathbb{Z}}\otimes_{\mathbb{Z}}\mathbb{C}$ its complexification. Fix a number $\mathbb{Z}$ and

\begin{defn}\leavevmode
	A \textit{\textbf{Hodge structure of weight $k$}} is the following data
	\[k:V_\mathbb{Z}\rightsquigarrow V_{\mathbb{C}}=\bigoplus_{p+q=}  V^{p,q}\]
	with
	\[V^{p,q}=\overline{V^{p,q}}\]
\end{defn}

A Hodge structure on $V$ is equippied with a $\mathsf{U}(1)$-action, with $z\in\mathsf{U}(1)$acting as $z^{p-q}$ on $V^{p,q}$.

\begin{defn}\leavevmode
	Let $V^*$ be a Hodge structure of wieght $k$. Let
	\[Q:V_\mathbb{Z}^k\times V_{\mathbb{Z}}^k\to  \mathbb{Z}\]
	be a $(-1)^{k}$-symmetric bilinear form wuch that its $\mathbb{C}$-bilinear extension to $V_\mathbb{C}$
	\begin{itemize}
	\item $Q(u,v)=(-1)^{k} Q(v,u)$.
	\item $Q(u,v)=0$ for  $u\in V^{p,q}$, $v\in V^{a,b}$, where $p\neq b$ and $q\neq a$.
	\item The form $(\sqrt{-1})^{p-q} Q(u,\bar{u})$ is positive definite on the space $V^{p,q}$.
	\end{itemize}
\end{defn}

Now let $X$ be a projective manifold (we need projective for polarization).
\[V^k\rightsquigarrow H^{k}(X,\mathbb{Z})=H^{p,q}(X)\]
To define a polarization you have to take a primitive component 
\[H^k_{\operatorname{pr i m i ti v e}}(X^n)=\ker\left( \begin{aligned}
	H^k  &\longrightarrow H^{2k-k+2} \\
	a &\longmapsto a\wedge \omega^{n-k+1}
\end{aligned} \right) \]
where $ \omega$ is the Kähler form.

Now
\[H^{k}(X)=\bigoplus_{i\geq 0} \omega^i\wedge H^{k-2i}_{\operatorname{primitive}}(X) \]
and
\[Q(u,v):=(-1)^{k(k-1)} \int x\wedge y\wedge \omega^{n-k}\]

\begin{defn}\leavevmode
	A \textit{\textbf{holomorphic family}} is
	 \[f:\mathcal{X}\to B\]
	 submersive (surjective differentials), holomorphic map. Also assume it is proper.
\end{defn}

So we have a familty and the cohomology of the fibers form a local sysem $f:\mathcal{X}\to  B\rightsquigarrow H^{k}(X_{t\in B},\mathbb{Z})$.

Now suppose you have $\mathbb{Z}$ locally constant sheaf on $B$. Define \[\mathbb{V}^k_\mathbb{Z}:=R^kf_*\mathbb{Z}\]

{\color{5}$\mathsf{OK}$ so a sheaf that the stalk at each point is cohomology of the fiber. You don't need derived categories for that.} 

Now we get some module
\[\mathbb{V}^k:=\dfrac{Rf_* \mathbb{Z}}{\text{torsion} }\cong \mathbb{Z}^n\]
complexify
\[\mathbb{V}^k_{\mathbb{C}}=\mathbb{V}^k\otimes_{\mathbb{Z}}\mathbb{C}\]
The stalks admit a Hodge decomposition
\[(\mathbb{V}^k_{\mathbb{C}})_t=H^{k}(X_t,\mathbb{C})\overset{ddc}{=}\bigoplus_{p+q=k} H^{p,q}(X_t) \]

\begin{defn}\leavevmode
	\[\mathcal{V}^k=C_B^\infty\otimes_{C^\infty_B}\mathcal{V}^k\]
	is a vector bundle that comes with a connection called the \textit{\textbf{Gauss-Manin connection}} defined as follows
	\[\nabla :\mathcal{V}^k\to \Omega^{1}_B \otimes_{C^\infty_B}\mathcal{V}^k\]
	where $\Omega^1_B$ is a sheaf of smooth 1 forms on $B$.
\end{defn}

\section{Variation of Hodge structures}

Now assume that you have a holomorphic family $f:\mathcal{X}\to B$. Define a  \textit{\textbf{Variation of Hodge structure}} on  $\mathcal{X}$ as a complex vector bundle $\mathcal{V},\nabla )$ with a flat connection $\nabla$ such that

{\color{4}\begin{quotation}
	each point of the base you have fiber, Hodge decomposition,
\end{quotation}}
\[x\in B, \qquad V^k(x)=\bigoplus_{p+q=k}V^{p,q}  \]
{\color{4}\begin{quotation}
	so taking derivative will not take it too far,
\end{quotation}}
\[\nabla_{\xi^{1,0}}(V^{p,q} )\subset V^{p,q} \oplus V^{p-1,q+1}\]
\[\nabla_{\xi^{0,1} }(V^{p,q} \subset V^{p,q}\oplus V^{p+1,q-1},\]
where $\xi^{1,0}+\xi^{0,1} \in T^{1,0}_B\oplus T^{0,1}_B$ are the vector fields of types $(1,0)$ and  $(0,1)$. This is called  \textit{\textbf{Griffiths transversality condition}}.

{\color{4}\begin{quotation}
	So a variation of Hodge structure is complex bundle with connection satisfizng Griffiths transversality condition
\end{quotation}}

\begin{defn}\leavevmode
	A \textit{\textbf{polarized VHS}} is  $V^k=\bigoplus_{p+q=k}V^{p,q}  $ decomposition such that $\nabla$ preserves the polarization and the integer of rational lattice.
\end{defn}

\section{Monodromy and the theorem of fixed part}

\begin{defn}\leavevmode
	Let $\mathbb{V}^k$ be a Hodge structure of weight $k$, $B$ its base and $t\in B$. The \textit{\textbf{monodromy representation}} is just the map
	\[\rho:\pi_{1}(B,t) \to \mathsf{GL}(\mathbb{V}^k_t)\]
	The image $\Gamma=\rho(\pi_{1}(B,t) )\subseteq \mathsf{GL}(\mathbb{V}^k_t,\mathbb{Z})$ is called the \textit{\textbf{monodromy group}}.
\end{defn}

There's a very nice
\begin{thm}[Deligne]\leavevmode
	Let $B$ be be a smooth quasi projective variety, $\mathbb{V}$ a polarized VHS on $B$ with the trivial monodromy. Then the VHS $\mathbb{V}$ is trivial.
\end{thm}

\begin{thm}\leavevmode
	Let $(X,I,J,K)$ be a compact hypercomplex manifold. Then  $\operatorname{Tw}(X)$ cannot be Moishezon.
\end{thm}

\begin{proof}\leavevmode
	Ad absurdum. Assume $\operatorname{Tw}(X)$ is Moishezon. 

	\begin{enumerate}[label=\textbf{Step \arabic*}]
		\item The Hodge-to-de-Rham spectral sequence of Moishezon manifolds degenerates in $E_1$:
\[E^{p,q}_1=H^{q}(X,\Omega^p)\implies H^{p+q}(X)\]
\begin{remark}[Misha]\leavevmode
	(As I understand…)	To get $H^{p,q}(X)=H^{q,p}(X)$ you need ddc lemma, doesn't follow only from spectral sequence. In fact you don't need spectral sequence. In fact that makes it harder.
\end{remark}

\begin{remark}[Mitia]\leavevmode
	If you have ddbar lemma, you have Hodge decomposition.
\end{remark}
	This defines a VHS over $\mathbb{C}P^{1}$.

\begin{remark}[André]\leavevmode
	You have decomposition on fibers, but you need more for VHS.
\end{remark}

$\mathsf{OK}$ so just suppose that the Moishezon condition allows for a VHS.

\item 
\[\begin{tikzcd}
\widetilde{\operatorname{Tw}(X)}\text{ projective} \arrow[d,"\mu\text{ bimeromorphic} "]\arrow[dd,bend right,swap,"\tilde{\pi}:=\pi\circ \mu"]\\
\operatorname{Tw}(X)\arrow[d,"\pi\text{ hol. submersion} "]\\
\mathbb{C}P^{1}
\end{tikzcd}\]
$\mathsf{OK}$ so the first one is projective so it has Fubini-Study metric. Sard's theorem says $\tilde{\pi}$ is well-defined everywhere. {\color{2}The bimeromorphic $\mu$ induces injections on cohomologies. This makes VHS downstairs inject into VHS upstairs.} This is clear if you believe in projection formula (you have to be a believer).

So we get a polarized VHS
\[\begin{tikzcd}
\operatorname{Tw}(X)\arrow[d]\\
\mathbb{C}P^{1}
\end{tikzcd}\]

\item Now by Deligne's theorem we just need to show that this VHS is non-trivial to get a contraditiction. (It should be trivial, so looks like monodromy is trivial.)

\item Let $X_I=(X,I)=\pi^{-1}(I)$, $\pi:\operatorname{Tw}(X)\to \mathbb{C}P^{1}$. Assume that $H^{1}(X_I)\neq 0$. $(X,I)$ and $(X,-I)$ be the fibers of  $\pi$, $\alpha\in H^{1,0}_I$
	\[I\alpha=\sqrt{-1}\alpha=-(-\sqrt{-1}\alpha)=-(-1\alpha)\implies \alpha\in H^{0,1}(X)\]
{\color{4}\begin{quotation}
	Something holomorphic with respect to $I$ is anti-holomorphic with respect to $-I$. Because you have $I$ and  $-I$ in the same quaternionic structure.
\end{quotation}}

So the VHS is non-trivial.

Now assume $H^{1}(X_I)=0$. Then
\[H^{0,1}(X_I)=0=H^{1}(X_I,\mathcal{O}_X)\]
\[\begin{tikzcd}[column sep=small]
	\cdots\arrow[r]&H^{1}(X_I,\mathcal{O}_{X_I})=0\arrow[r]&!\arrow[r]&\arrow[r]&\arrow[r]&\arrow[r]&\cdots
\end{tikzcd}\]
So any topologically trivial bundle is also holomorphically trivial.

\item Now we consider the \textit{middle cohomology} of the fiber $X_I$.Let $0\neq \phi\in\Omega^{n}(X_I)$ Consider a VHS associated with the middle cohomology
	\[H^{2n,0}(X)=\left<\phi_I\right> \]
However, the fiberwise canonical bundle of the twistor space is isomorphic to a guy, that is,
	\[\Omega^{2n}_\pi(\operatorname{Tw}(X))\cong \mathcal{O}(-2n)\]
and the latter has no global sections. That's a contradiction.


\end{enumerate}\end{proof}

\end{document}
