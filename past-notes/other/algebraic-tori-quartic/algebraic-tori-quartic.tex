\input{/Users/daniel/github/config/preamble.sty}%available at github.com/danimalabares/config

\begin{document}

\begin{minipage}{\textwidth}
	\begin{minipage}{1\textwidth}
		Semin\'ario das Sextas \hfill PUC-Rio
		
		{\small\hfill\href{https://github.com/Friday-seminar/}{github.com/Friday-seminar}}
	\end{minipage}
\end{minipage}\vspace{.2cm}\hrule

\vspace{10pt}

{\Huge Algebraic tori in the complement of quartic surfaces}

\vspace{1em}
22 Janeiro

\hfill{\Large Eduardo Alvez da Silva}

\hfill{\large Institut de Mathématiques d'Orsay}

\tableofcontents

\begin{enumerate}
\item Recall from Eduardo's last talk that

\begin{defn}
	A \textit{\textbf{log Calabi-Yau}} pair is a lc pair  $(X,D)$ consisting of a normal projective variety $X$ and a reduced Weil divisor $D$ such that $K_X+D\sim_{\mathbb{Z}}0$.
\end{defn}

Now we have given a slightly different definition.

\item  It was possible to classify Fano threefolds using these methods. (Recent.) Why? The anticanonical bundle is ample so there is a power that gives a section…
\item 
	\begin{thm}[Gross-Hacking-Keel, 2017]\leavevmode
	Log CY pair of index 1 and of corregularity 0 is log rational.
	\end{thm}
	\begin{thing5}{Conjecture}[Shokirov]\leavevmode
	Every threefold rational log CY pair \((X,B )\) and corregularity 0 is log rational.
	\end{thing5}
\item  Perhaps this has to do with a classification of toric varieties:
	\begin{thm}[Brown, McKerman, Svaldi, Zong, 2018]\leavevmode
	\((X,B)\) log CY pair implies \(c (X,B ) \geq 0\).
	\end{thm}
No I think it's this one:
\begin{thing5}{Idea}\leavevmode
If \(c(X,B)<1\), then \(X,\left\lfloor B \right\rfloor\) is toric.
\end{thing5}
\item Toric \(\implies \) cluster type \(\implies \) log rational. Counter examples for both of the reverse arrows are known.
\item 
	\begin{thm}[-,Figueroa, Moraga, 2024]\leavevmode
	\((\mathbb{P}^3,B)\) log CY pair, \(i=1\) corregularity 0. Assume Bb is non-normal. \((\mathbb{P}^3, B )\) is cluster type unless the nodal locus of  \(B\) is centered in a plane.
	\end{thm}
\end{enumerate}
\end{document}
