\input{/Users/daniel/github/config/preamble.sty}%available at github.com/danimalabares/config
\input{/Users/daniel/github/config/thms-eng.sty}%available at github.com/danimalabares/config

\begin{document}

\begin{minipage}{\textwidth}
	\begin{minipage}{1\textwidth}
		Semin\'ario das Sextas \hfill PUC-Rio
		
		{\small\href{https://github.com/Friday-seminar/}{github.com/Friday-seminar}\hfill\href{https://github.com/danimalabares/seminars}{github.com/danimalabares/seminars}}
		\end{minipage}
\end{minipage}\vspace{.2cm}\hrule

\vspace{10pt}

{\Huge Automorphisms of hyperkahler manifolds and fractal geometry of hyperbolic groups}


\hfill{\Large Misha Vertbitsky}

\hfill{\Large IMPA}

\hfill{\large December 6 2024}

\[\mathbb{H}^n=\frac{\mathsf{SO}^+(1,n)}{\mathsf{SO}(n)}=\mathbb{P}^+(\mathbb{R}^{1,n})\]
Kleinian groups: $\Gamma \subset \operatorname{Is o}(\mathbb{H}^n)$ of finite covolume.

$\mathbb{H}^n/\Gamma$ is a hyperbolic orbifold.

The \textit{\textbf{absolute}} is the projectivization of the vanishing-locus of the quadratic form. $\operatorname{A bs }=\mathbb{P}^0(\mathbb{R}^{1,n}=\partial\mathbb{H}^n=S^{n-1}$.


Thm: Harish-Chandra: for any lattice of signature $1,n$,  $\mathsf{SO}(\Lambda)$ is Kleinian, i.e. has finite covolume.

\section{CHOPB}

A CHOPB is a connected component of $\mathcal{H}\setminus \bigcup S_i $ where $\mathcal{H}=\mathbb{H}^n/\Gamma$ and $S_i$ are hypersurfaces. It is a convex polyhedron. $\pi_1(P)=\operatorname{St}_P(\Gamma)$. It will be the automorphism group of things like K3 surfaces.

The \textit{\textbf{absolute boundary}} is $\overline{P}\cap \operatorname{Ab s}$. It is the limit set of $\operatorname{S t}_P\Gamma$. I think the limit set is the intersection of the closure of a set with the absolute.

Equivalently, every lattice can be realised as the Picard lattice of a K3 surface (Nikulin, 80s).

\begin{upshot}\leavevmode
	From a lattice we produce a CHOPB.
\end{upshot}


\section{Relation to Kähler structures}

The complex moduli is an algebraic variety. The mirror symmetry predics that the complex moduli would correspond with the Kahler moduli. But there is no algebraic variety associated with the Kähler cone.

\textbf{Related to cone conjecture}: Consider only the imaginary part of the Kähler cone. Then add $ H^{11}(M)$. 




\end{document}
