\input{/Users/daniel/github/config/preamble.sty}%available at github.com/danimalabares/config
\input{/Users/daniel/github/config/thms-eng.sty}%available at github.com/danimalabares/config

\begin{document}

\begin{minipage}{\textwidth}
	\begin{minipage}{1\textwidth}
		Semin\'ario das Sextas \hfill PUC-Rio
		
		{\small\href{https://github.com/Friday-seminar/}{github.com/Friday-seminar}\hfill\href{https://github.com/danimalabares/seminars}{github.com/danimalabares/seminars}}
		\end{minipage}
\end{minipage}\vspace{.2cm}\hrule

\vspace{10pt}

{\Huge The homotopy type of spaces of locally convex curves in the sphere}

\hfill{\Large Nicolau Saldanha}

\hfill{\Large PUC-Rio}

\hfill{\large 22 November 2024}

\begin{thing4}{Abstract}
	A smooth curve $\gamma:[0,1]\to S^2$ is locally convex if its geodesic curvature is positiva at every point. J. A. Little shows the the space of all locally positive curves $\gamma$ with $\gamma(0)=\gamma(1)=e_1$ and $\gamma'(0)=\gamma'(1)=e_2$ has three connected components $L_{-1}$, $L_{+1}$, $L_{-1,n}$. These spaces and variants have been discussed, among others, by B. Shapiro, M. Shapiro and B. Khesin.

	Our first aim is to describe the homotopy type of these spaces. The connected component $L_{-1,c}$ is known to be contractible. We construc maps from $L_{+1}$ and $L_{-1,n}$ to $\Omega S^3 \vee S^2 \vee S^6 \vee S^{10}\vee \ldots$ and $\Omega S^3\vee S^4 \vee S^8 \vee S^{12} \vee \ldots$, respectiely, and show thtat they are (weak) homotopy equivalences.

	More generally, a smooth curve $\gamma:[0,1]\to S^n \subset \mathbb{R}^{n+1}$ is \textit{\textbf{locally convex}} if $\det(\gamma(t),\ldots,\gamma^n(t))>0$ for all $t$. We would like to understand the motopy type of thte space $L$ of locally convex curves with $\gamma^{(j)}(0)=\gamma^{(j)}(1)=e_{j+1}$ for all $j \neq n$. We describe a CW complex with the same homotopy type. The homotopy type ogf $L$ is described for $n=3$.
\end{thing4}

\section{Introduction}


We have started with a \textit{\textbf{locally convex}} curve put in $S^3$ and then pass to a curve in $\mathsf{SO}(3)$ by taking the Senet-Ferret frame. Also generalize this to $n$ dimensions by taking the matrix  $(\gamma(t),\dot\gamma(t),\ldots, \gamma^{(n)}(t))$ (det of this matrix is not zero is definition locally convex) and to other spaces like $\mathbb{P}^n$.

We can start with a curve that is not locally convex but take a phone wire curve near and make it locally convex (though in the lie group the two things may not be close together).

Then consider for $Q \in \mathsf{SO}(3)$, let $\mathcal{L}_{Q}$ be the space of locally convex curves $\gamma$ with $\mathcal{F}_\gamma(0)=\operatorname{Id}$and $\mathcal{F}_\gamma(1)=Q$. We want to look at loops $ \mathcal{L}(\operatorname{Id})$.

The objective of this talk is to describe this space: cohomology, connected components…?

\section{Connected components of $\mathcal{L}_Q$}

Recall that $\pi_{1}(\mathsf{SO}(n+1))=\mathbb{Z}/2$. Lifting to the universal cover $\mathsf{Spin}(n+1)$ guarantees that $\mathcal{L}_Q$ has at least two connected components. For $n=2$ this corresponds to checking the parity of the intersection with hyperplanes ({\color{3}this reminds me of algebraic degree…})


\end{document}
