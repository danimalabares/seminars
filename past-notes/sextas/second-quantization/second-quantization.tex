\input{/Users/daniel/github/config/preamble.sty}%available at github.com/danimalabares/config
\input{/Users/daniel/github/config/thms-eng.sty}%available at github.com/danimalabares/config

\begin{document}

\begin{minipage}{\textwidth}
	\begin{minipage}{1\textwidth}
		Semin\'ario das Sextas \hfill PUC-Rio
		
		{\small\href{https://github.com/Friday-seminar/}{github.com/Friday-seminar}\hfill\href{https://github.com/danimalabares/seminars}{github.com/danimalabares/seminars}}
		\end{minipage}
\end{minipage}\vspace{.2cm}\hrule

\vspace{10pt}

{\Huge Second Quantization}

\hfill{\Large Altan Erdnigor}

\hfill{\Large IMPA}

\hfill{\large 18 October 2024}
\begin{thing7}{Abstract}\leavevmode
	The talk will cover a survey by Yuri Neretin of a famous eponymous book by Felix Berezin. Neretin's overview is short and clean, with no proofs (they are all exercises in functional analysis), providing an abundance of references.

	The topics are: finite-dimensional and infinite-dimensional bosonic Fock space ( \href{https://en.wikipedia.org/wiki/Segal%E2%80%93Bargmann_space}{Segal-Bargmann space} (1961, 1963)) and fermionic Fock spaces, and the Weil representation of the orthogonal and symplectic groups.

The main theorem of the paper is that the representations do exist. Mostly it's about giving the right definitions. If we don't discuss the proofs, I can fit it under one hour.
\end{thing7}

\tableofcontents

\section{Plan}
\begin{enumerate}
	\item Bossonic Fock space $F_n\cong L^2(\mathbb{R}^n)$. What is it, how to work with it. This will be the quantization of ordinary space $\mathbb{R}^{2}$.

	\item Metaplectic representation (Weil representation) $\mathsf{Sp}(\mathbb{R})$ (symplectic matrices). The projective action $\mathsf{Sp}_{2n}(\mathbb{R})\curvearrowright F_n$. You will see how strange it is that this representation {\color{4}?}

	\item Fermionic Fock space $\Lambda_n\cong \Lambda^\bullet(\mathbb{C}^n)$.

	\item Spinor represenation. Fermionic Fock space also admits a projective representation called spinor represenation: $\mathsf{SO}(2n) \curvearrowright \Lambda_n$.
	
	\item $n\to \infty$. Gives rise to $F_\infty, \Lambda_\infty$.
\end{enumerate}

Big formulas give rise to representations.


\section{Fock space}

Consider $\mathbb{C}^{n}$. define
\[F_n=\left\{\text{$f(z)$-entire functions  $\mathbb{C}^{n}\to \mathbb{C}$}\Big| \int_{\mathbb{C}^{n}}(f(x))^2e^{-|z|^2}d\lambda(z)<\infty\right\}\]

We have an inner product:
\[\left<f,g\right> :=\int_{\mathbb{C}^{n}}f(z)\overline{g(z)}e^{|z|^2}d \lambda(z)\]

\begin{prop}\leavevmode
	$z^k:=z_1^{k_1}\cdot\ldots\cdot z_n^{k_n}$, where $k=(k_1,\ldots,k_n)$, form an orthogonal basis (of monomials) for $F_n$. Thus $F_n$ is a Hilbert space. Moreover, $\|z^k\|^2=k_1!\ldots k_n!$.
\end{prop}

\begin{question}\leavevmode
	Why is this basis orthogonal--- why do two different monomials integrate to zero?
\end{question}

\begin{remark}\leavevmode
	See \textit{Harmonica analysis in phase space} by Folland (1988).
\end{remark}

\begin{defn}\leavevmode
	$b_T(z):=\operatorname{exp}(zTz^+)$ for $T$ symmetric.
\end{defn}

\begin{claim}\leavevmode
	$b_T\in F_n \iff \|T\|<1$.
\end{claim}

\begin{exercise}\leavevmode
	$\left<b_T,b_S\right> =\det ( (1-TS^*)^{-\frac{1}{2}} )$ \textbf{Hint:} Appendix 1 in Folland. This implies that
	\[\|b_T\|=\det (1-T T^* )^{-\frac{1}{4}}=\sqrt{\det (1-T T^*)^{-\frac{1}{2}}} \]
\end{exercise}
\clearpage

{\Large Altan's talk exercise 2: \\Fock space is reproducing kernel space}

\begin{defn}[Coherent states]\leavevmode
	For $a\in \mathbb{C}^{n}$,
	\[\varphi_a(z):=\operatorname{exp}(z_1\overline{a_1}+\ldots +z_n\overline{a_n})\]
\end{defn}

\begin{claim}\leavevmode
	For $f\in F_n$,
	\[\left<f,\varphi_a\right> =f(a)\]
\end{claim}

\begin{remark}\leavevmode
	This says that for every $a\in\mathbb{C}^{n}$ the functional $F_n\ni f\mapsto f(z)\in\mathbb{C}$ is bounded since it is represented by $\varphi_a$ (Riesz representation theorem). This is called \textit{\textbf{Reproducing Kernel Hilbert Space}}.
\end{remark}

\begin{proof}[Proof (Exercise)]\leavevmode
	Let's consider the case $n=1$. Then  $\varphi_a=\operatorname{exp}(z\bar{a})$ and
	\[\left<f,\varphi_a\right> =\int_{\mathbb{C}}f(z)\overline{\operatorname{exp}(za) }e^{-|z|^2}d \lambda(z)\]
		{\color{2}But I'm not sure how to compute this integral. Upon difficulties let's suppose $f\equiv1$. Perhaps split complex exponential into real and imaginary parts? Then we get real-valued integrals over $\mathbb{R}^{2}=\mathbb{C}$:}
		\begin{align*}
		\int_{\mathbb{C}}\overline{\operatorname{exp}(za)} e^{-|z|^2}d \lambda(z)&=\int_{\mathbb{R}^{2}}\overline{e^{\operatorname{Re}(za)}\big(\cos \operatorname{Im}(za)+i \operatorname{Im}(za)\big)}e^{-|z|^2}d \lambda(z)\\
										     &= \int_{\mathbb{R}^{2}}\overline{e^{\operatorname{Re}(za)}\cos \operatorname{Im}(za)}e^{-|z|^2}d \lambda(z)+\int_{\mathbb{R}^{2}}\overline{e^{\operatorname{Re}(za)}i \operatorname{Im}(za)}e^{-|z|^2}d \lambda(z)
			\end{align*}
{\color{2}Not sure…}

	In the general case we shall have
	\begin{align*}\left<f,\varphi_a\right> &=\int_{\mathbb{C}^n}f(z)\overline{\varphi_a(z)}e^{-|z|^2}d\lambda(z)\\
		&=\int_{\mathbb{C}^{n}}f(z)\overline{\operatorname{exp}(z_1\overline{a}_1+\ldots +z_n\overline{a}_n)}e^{-|z|^2}d \lambda(z)
		\end{align*}
{\color{2}Maybe take this somehow to a line integral around $a$ to Cauchy integral formula…?} \end{proof}
\clearpage
\begin{remark}\leavevmode
	Evaluation at a point is a bounded functional (this is not the case in general  $L^2$ space because you can have functions with arbitrarily large values).
\end{remark}

\begin{question}\leavevmode
	What are the operators $A\curvearrowright F_n$?
\end{question}

\begin{thing5}{Answer}\leavevmode
	$(\operatorname{Af})z=\int_{\mathbb{C}^{n}}k(z,\bar{u})f(u)e^{-|u|^2} d \lambda(u)$, where $K(z,u)$ is holomorphic in  $z$ and antiholomorphic in $u$. The integral converges absolutely for all $f \in F_n$.
\end{thing5}

\begin{proof}\leavevmode
	We need to compute the kernel. So define
	\[c_ke=\left<Az^{\mathbf{k}},z^\ell\right> \]
	then
	\[k(z,\bar{u} )=\sum_{k,\ell}c_{k\ell}\frac{z^k}{k!}\frac{\bar{u} ^\ell}{\ell!}\]
	which means that the kernel is symmetric. So this means that restricting the kernel to diagonal is not holomorphic anymore, kind of real analytic (?).

	Alternatively, $k(a,b)=\left<A\varphi_b,\varphi_a\right> $.
\end{proof}

\begin{question}[Dani]\leavevmode
	Why do we need to compute the kernel? We want to show that the operator $A\curvearrowright F_n$ produces a (projective) action, right?
\end{question}

\begin{thing7}{$\mathsf{OK}$}\leavevmode
	so we defined Fock space and the operators that act nicely on it.
\end{thing7}


\section{Metaplectic representation}

\begin{defn}\leavevmode
	\[\mathsf{Sp}(2n,\mathbb{R}:=\left( h\in\operatorname{Mat}_{2n\times 2n}:h \begin{pmatrix} 0&1\\-1&0 \end{pmatrix}h^+=\begin{pmatrix} 0&1\\-1&0 \end{pmatrix}   \right) \]
\end{defn}

Let
\[J=\frac{1}{\sqrt{2} }\begin{pmatrix} 1&i\\i&1 \end{pmatrix} , \qquad g:=Jg J^{-1}\]\[g=\begin{pmatrix} Q&\psi\\\bar{\psi} &\bar{\phi}  \end{pmatrix} ,\qquad g \begin{pmatrix} 1&0\\0&-1 \end{pmatrix} g^* =\begin{pmatrix} 1&0\\0&-1 \end{pmatrix} \]

Also 
\[\mathsf{Sp}(2n,\mathbb{R})\cong \mathsf{U}(n,n) \cap \mathsf{Sp}(2n,\mathbb{C})\]
Now
\[\left(W\begin{pmatrix} \phi&\psi\\\bar{\psi} &\bar{\phi}  \end{pmatrix} f\right)(z)=\int_{\mathbb{C}^{n}}\operatorname{exp} \left\{ \frac{1}{2}(z,\bar{u} )\begin{pmatrix} \bar{\psi} \phi^{-1}&(\phi^t)^{-1}\\\phi^{-1}&-\phi^{-1}\psi \end{pmatrix} \begin{pmatrix} z^t\\\bar{u} ^t \end{pmatrix}  \right\} f(u)e^{-|u|^2}d \lambda(u).\]

{\color{8}Who is this guy?} It's a \textit{\textbf{Berezin's integral}}. So that is the definition of $W$, which is the representation of the mataplectic group $W:\mathsf{Sp}_{2n}(\mathbb{R})\curvearrowright F_n$ we wanted.

Now let's explain what is a projective representation.

\begin{thm}\leavevmode
\begin{enumerate}

	\item 
$W(\cdot )$ are unitary up to scalar. More precisely,
	\[\det (\phi^*\phi)^{-\frac{1}{4}} W \begin{pmatrix} \phi&\psi\\\bar{\psi} &\bar{\phi}  \end{pmatrix} \]
	are unitary.

	\item $W(\cdot )$ define a projective representation of $\mathsf{Sp}_{2n}(\mathbb{R})\curvearrowright F_n$.

		Now let's finish with this nice formula for the cocycle (how to multiply these operators?:

	\[W \begin{pmatrix} \phi_1 &\psi_1\\\bar{\psi}_1 &\bar{\phi}_1 \end{pmatrix}W \begin{pmatrix} \phi_2&\psi_2\\\bar{\psi}_2&\bar{\phi}_2  \end{pmatrix} =\det (1+\phi_1^{-1}\psi_1\bar{\psi}_2\phi_2^{-1})^{-\frac{1}{2}} W\left( \begin{pmatrix} \phi_1&\psi_1\\\bar{\psi}_1&\bar{\phi}_1   \end{pmatrix}  \right)  \]
\end{enumerate}	\end{thm}

\begin{question}[Dani]\leavevmode
	So what is projective representation?
\end{question}













\end{document}
