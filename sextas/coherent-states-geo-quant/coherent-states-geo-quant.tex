\input{/Users/daniel/github/config/preamble.sty}%available at github.com/danimalabares/config
\input{/Users/daniel/github/config/thms-eng.sty}%available at github.com/danimalabares/config

\begin{document}

\begin{minipage}{\textwidth}
	\begin{minipage}{1\textwidth}
		Semin\'ario das Sextas \hfill PUC-Rio
		
		{\small\href{https://github.com/Friday-seminar/}{github.com/Friday-seminar}\hfill\href{https://github.com/danimalabares/seminars}{github.com/danimalabares/seminars}}
		\end{minipage}
\end{minipage}\vspace{.2cm}\hrule

\vspace{10pt}

{\Huge Geometric quantization on Kähler manifolds, Berezin-Toeplitz, coherent states}

\hfill{\Large Bruno Suassuna}

\hfill{\Large PUC-Rio}

\hfill{\large 18 October 2024}

The idea of geometric quantization is to start with a symplectic manifold $(M,\omega)$ and for some geometric structure that you have on $M$, like a line bundle with a hermitian metric, which is the geometric quantization data. 

So on every point $x \in M$ you have a copy of $\mathbb{C}$, and you put a hermitian metric $h_x:\mathcal{L}_x\otimes \bar{\mathcal{L}}_x\to \mathbb{C} $.

Now the second part of the geometric quantization data is a connection $\nabla$ on $\mathcal{L}$ that preserves $h$. Recall that this means that
\[\nabla :\mathcal{L}\to \mathcal{L}\otimes \Omega^{1}(M)\]
such that
\begin{itemize}
\item $\forall v\in\mathcal{C}^\infty(TM)$, $\nabla_v:\mathcal{L}\to  \mathcal{L}$ is $\mathbb{C}$-linear
\item Leibniz in the sense that $\nabla_v(fs)=df(v)s+f \nabla_vs$.
\end{itemize}

\begin{thing6}{Property}[Bohr-Sommerfield]\leavevmode
	$\frac{2\pi}{i}R_\nabla =\omega$, where $(E,\nabla )$ is a vector bundle with connection we define $R_\nabla \in\Omega^{2}_M \otimes \operatorname{End}(E)$ (the curvature).
\end{thing6}

\begin{defn}\leavevmode
	$M$ is \textit{\textbf{quantizable}} when it admits the data above satisfyint the property Bohr-Sommerfield. 

	We call $L^2(M,\mathcal{L})$ a \textit{\textbf{pre-quantum Hilbert space}}.
\end{defn}

\begin{question}\leavevmode
	What is it that makes it better in Kähler manifolds? (Dani: I thought in general the function space didn't have to be the sections of a bundle)
\end{question}

\begin{remark}[Sergey]\leavevmode
	When $\mathbb{R}^{2n}$ is the phase space and $\mathcal{L}$ is the trivial bundle then $L^2( \mathbb{R}^{2} )$ is \textit{too big!} 
\end{remark}

\begin{defn}\leavevmode
	A \textit{\textbf{Kähler manifold} } is $(M, \omega,g,I)$ where $\omega$ is a symplectic structure, $g$ a riemannian structure and  $I$ a complex structure, and they are all compatible and $I$ is integrable. So remember that compatibility is for example that $g(u,v)=\omega(Iu,v)$ and $g$ is  $I$-invariant. On the other hand, integrability is a non-trivial PDE and gives holomorphic local coordinates.
\end{defn}

{\color{3}\bfseries Starting again,}\hspace{.5em} take a Kähler manifold $M$ and define a pre-quantum line bundle so  $\mathcal{L}$ holomorphic with the Chern connection $\nabla$. Also suppose that $\nabla$ satisfies the Bohr-Sommerfield property.

\begin{remark}[Kodaira theorem]\leavevmode
	The Kähler class (cohomology class of symplectic form) is the first Chern class, i.e., $[\omega]=c_1(\ell)\in H^{2}(M,\mathbb{Z})$ iff $\mathcal{L}$ is ample.
\end{remark}

\begin{defn}\leavevmode
	\[\mathcal{H}_m=H^{0}(X,\mathcal{L}^{\otimes m})\]
	so the sections of that bundle.
\end{defn}

You see hare making the space of sections smaller:
	\[\mathcal{H}_m=H^{0}(X,\mathcal{L}^{\otimes m})\hookrightarrow L^2(M,\mathcal{L}^{\otimes m})\]
And if $M$ is not compact, take
\[H^{0}(M,\mathcal{L}^{\otimes m})\cap L_2\]

\begin{question}[Altan]\leavevmode
	What about the measure here?
\end{question}

\begin{thing9}{Answer}[Bruno]\leavevmode
	Related to Liouville measure; a clever choice of $h$. So maybe
	 \[\left<s_1,s_2\right> =\int_{M}h\left( s_1(x),s_2(x) \right) \frac{\omega^d}{d!}\]
	 And now $d$ is what in Altan's talk was  $n$… or was it $m$?
\end{thing9}

\begin{remark}\leavevmode
	So in the case of Altan's talk
	\[\mathcal{H}=\left\{ f:\mathbb{C}^{n}\overset{\operatorname{entire}}{\longrightarrow}\mathbb{C}:\int_{\mathbb{C}^{n}}|f|^2\operatorname{exp}(-|z|^2) d \lambda(z) \right\} \]
	so notice that the $\operatorname{exp}$ term is stopping the space to be trivial because bounded holomorphic functions are constant right?
\end{remark}

$\mathsf{OK}$ back to the quantization. Given $f \in\mathcal{C}^\infty(M)$ define $A_f \in\operatorname{End}(\mathcal{H}_m)$…

\begin{thing4}{Berezin-Toeplitz}\leavevmode
\[T_f(S)=\Pi_m(fS) \]
$\Pi_m$ orthogonal projection to $H^{0}(M,\mathcal{L}^{\otimes m})$.
\end{thing4}

\begin{thing7}{Kostant-Sorian pre-quantum operators}\leavevmode
	$f$ quantizable means $Q_f(s):=\nabla_{X_f}s-2\pi i fs$. Here $X_f$ is the hamiltonian vector field of  $f$ with respect to  $\omega$. 

	So that's an operator associated to a function---quantization!

	And we \textit{wish} that if  $s$ is holomorphic then so is $Q_f(s)$.
\end{thing7}

\begin{remark}[Sergey]\leavevmode
	Also in the wish list is some property of brackets right?
\end{remark}

Right so we are doing
\begin{align*}
	\mathcal{C}^\infty(M) &\longrightarrow \operatorname{End}(V) \\
	f &\longmapsto \hat{f}
\end{align*}
and we want
\[ \hat{f},\hat{g}]=i\hbar \widehat{\{f,g\}}\qquad \hat{1}=\operatorname{Id},\qquad \widehat{\varphi\circ f}=\varphi(\hat{f})\]

\begin{question}[Dani]\leavevmode
	So why is it impossible that the wishes became true? Is that a difficult theorem?
\end{question}

Now 
\[L_f(s)=\Pi_m(Q_f(s))\]
and
\begin{thing9}{Tuyman's lemma}\leavevmode
	$iQ_f=T_f-\perp_{2m}\Delta f$
\end{thing9}

\begin{thing8}{Coherent state quantization}\leavevmode
	$ p \in \mathcal{L}_x^{\otimes m}\setminus \{0\} \rightsquigarrow e_p \in \mathcal{H}_m$. So
	\[s \in H^{0}(X,\mathcal{L}^{\otimes m})\longmapsto s(x)\in\mathcal{L}_x=\lambda_p(s)\cdot p\]
	Now a \textit{\textbf{coherent state}} is defined via Riesz' representation theorem so  $\lambda_p \in\mathcal{H}_m^*$ such that
	\[\lambda_p(s)=\left<e_p,s\right> \]
\end{thing8}

\begin{question}[Dani]\leavevmode
	Why are coherent states so important?
\end{question}

They give us a rational map which is essentially Kodaira map (the map defined by any line bundle)
\[\operatorname{coh}:M\dashrightarrow \mathbb{P}(\overline{\mathcal{H}}_m\]
So given a bounded operator $A\in B(\mathcal{H}_m)$ we define \[\hat{A}(x)=\frac{\left<Ae_p,e_p\right> }{\|e_p\|^2}\qquad  x\in M\]
called the \textit{\textbf{covariant symbol}}.  {\color{7}(That is not the same hat than the hat of the quantization!)} We have a map $B (\mathcal{H}_m)\to  \mathcal{C}^\infty(M)$.

Now we try to invert this map: which functions are the covariant symbols of some bounded operators?

\begin{remark}[Sergey]\leavevmode
	This is like in altan talk, look for the kernel!
\end{remark}

Here's two formulas to go in the other direction:

\[(As)(x)=\int_{M}h_y(s(y),s(y))\hat{A}(x,y)\frac{\omega^m}{n!}\]
\[\operatorname{Tr}(A)=\int_{M}\hat{A}(x)\theta(x)\frac{\omega^n}{n!}\]
and that's a nice formula wich has to do with Rawsley, $\theta(x)=|q|^2\cdot\|e_q\|^2$.

\begin{remark}[Dani]\leavevmode
	So it looks like we want an equivalence of operators and functions.
\end{remark}

\begin{thing7}{Bondermann-Meinken-Schlihenmeier}\leavevmode
	$\|f\|_\infty-\frac{c}{m}\leq \|T_f^{m}\|\leq \|f\|_\infty$
	\[\|m[T_f^{(m)}, T_y^{(m)} ]iT_{\{f,g\}}^{(m)}\|= \mathcal{O}(m^{-1})\qquad \text{as }n\to \infty \]
\end{thing7}

\end{document}
