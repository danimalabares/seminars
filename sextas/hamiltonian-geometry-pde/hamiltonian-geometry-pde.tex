\input{/Users/daniel/github/config/preamble.sty}%available at github.com/danimalabares/config

\begin{document}

\begin{minipage}{\textwidth}
	\begin{minipage}{1\textwidth}
		Semin\'ario das Sextas \hfill PUC-Rio
		
		{\small\href{https://github.com/Friday-seminar/}{github.com/Friday-seminar}\hfill\href{https://github.com/danimalabares/seminars}{github.com/danimalabares/seminars}}
		\end{minipage}
\end{minipage}\vspace{.2cm}\hrule

\vspace{10pt}

{\Huge Some problems in Hamiltonian geometry of PDEs}

\hfill{\Large R. Vitolo}

\hfill{\Large Dipartimento di Matematica e Fisica 'E. De Giorgi'}

\hfill{\large 27 Septembro 2024}

\begin{idea7}{Abstract}\leavevmode
The Hamiltonian formulation of Partial Differential Equations is one of the cornerstones of the theory of Integrable Systems. Being carried out by analogy with finite-dimensional Hamiltonian systems, it has an intrinsic geometric nature. In this talk we will review differential-geometric aspects of the Hamiltonian theory of PDEs as well as new projective-geometric properties of known Integrable Systems that are emerging in recent years.	
\end{idea7}

\tableofcontents

\begin{itemize}
\item Introduction to the Hamiltonian formalism of PDEs
\item Geometry of the Hamiltonian formalism.
\item Classification of bi-Hamiltonian ?
\end{itemize}

%Why is momenta dual to velocity?

\begin{defn}
	$A$ is a \textit{\textbf{Hamiltonian operator}} if and only if
	\[ [F,G]=\int \frac{\delta G}{ \delta u^i}A^{ij}\frac{\delta}{\delta?}\]
\end{defn}

\begin{example}
KdV equation leads to a bi-Hamiltonian system.
\end{example}

\begin{itemize}
	\item Motivation for Hamiltonian PDEs: A Hamiltonian maps conservation laws to symmetries.

\item Classification of (?bi-)Hamiltonians leads to classification of PDEs.
\end{itemize}

\begin{remark}
	There is some sort of projective invariance of solutions of WDVV equation. Does that lead to some interesting result about solutions?
\end{remark}

Projective geometry of homogeneous second order Hamiltonian operators
Classification of linear operators making them correspond to algebraic varieties. See Vergallo-Vitolo, 2023, \textit{Projective geometry of homogeneous second order Hamiltonian operators} for a Fano variety found this way.

\begin{table}[H]
	\centering
	\begin{tabular}{c| c}
		3 order Hamilt op & quadratic line complex\\
	2 order & system of line compl\\
	$R_2$ first order &quadratic line complex
	\end{tabular}
\end{table}

\end{document}
