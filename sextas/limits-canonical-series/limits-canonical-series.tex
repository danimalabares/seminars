\input{/Users/daniel/github/config/preamble.sty}%available at github.com/danimalabares/config

\begin{document}

\begin{minipage}{\textwidth}
	\begin{minipage}{1\textwidth}
		Semin\'ario das Sextas \hfill PUC-Rio
		
		{\small\href{https://github.com/Friday-seminar/}{github.com/Friday-seminar}\hfill\href{https://github.com/danimalabares/seminars}{github.com/danimalabares/seminars}}
		\end{minipage}
\end{minipage}\vspace{.2cm}\hrule

\vspace{10pt}

Limits of canonical series for curves
{\Huge }

\hfill{\Large Eduardo Esteves}

\hfill{\Large IMPA}

\hfill{\large 20 Septembro 2024}

\begin{idea7}{Abstract}\leavevmode
	
\end{idea7}

\tableofcontents

\section{intro}

What is the canonical series? Consider $C$ a projective smooth connected curve over a closed field $k$. The cotangent bundle  $\omega_C$ is the same as the bundle of differentials, and the canonical bundle. It has complex dimension 1. The local sections are differentials.

A differential $\alpha \in\Gamma(C,\omega_C)$ is holomorphic, it can be meromorphic. It is written as $\alpha=fdt$ for some $f\in k(C)$. This leads to the notion of a \textit{\textbf{divisor}} of a differential, which in this case is
 \[\operatorname{div}(\alpha)=\sum \operatorname{ord}_p(t_p)p\]
 and it is also the \textit{\textbf{canonical divisor}}. 

 Now
 \[\omega_C=\mathcal{O}_C(k)\]
 The \textit{\textbf{genus}}, which is a topological invariant, is also  $\dim_k\Gamma(C,\omega_C)$.
 \[\operatorname{deg}(\omega_C)=\operatorname{deg}(K)=2g-2\]
 And
 \[\mathbb{H}=\Gamma(C,\omega_C)\]
 is the \textit{\textbf{canonical series}}.

  \begin{example}
 	Smooth quartic plane curve
	\[C=V(F)\subset \mathbb{P}^2\]
	We have that $\operatorname{deg}(F)=4$, and
	\begin{align*}
		\omega_C&=\mathcal{O}_C(1)=\mathcal{O}_{\mathbb{P}^2}(1)|_{C}\\
		g&=\dfrac{(d-1)(d-L)}{2}=3\\
		\mathcal{O}_C(1)&=\mathcal{O}_C(L\cap C)\\
		\operatorname{deg}(\mathcal{O}_C(1))&=4=2g-2\\
		\dim_k\Gamma(G\mathcal{O}_C(1))&=3=g
	\end{align*}
 \end{example}
\begin{quotation}
	Think of the canonical series as a space of linear sections of a line bundle, but also as a collection of divisors parametrized by $\mathbb{P}^2$
\end{quotation}

\begin{example}[A singular quadric]
\end{example}

\section{What we do}
Given a nodal curve $X$ (at a node there are two "branches" that intersect) which is general for its topology ($G=(V,E)$ dual graph) where
 \begin{itemize}
\item $V$ is the set of irreducible components). There is a correspondence of the vertices in this graph and curves in $X$ : $v\in V\iff X_v \subset X$.
\item $E$ is the set of nodes. Here $e \in E\iff N_e\in X$. So an edge is a pair of points if the node belongs to the intersection of the corresponding curves:
	\[e=\{u,v\} \iff N_e\in X_u\cap X_v\]
\item The \textit{\textbf{genus function}} associates to every component its geometric genus:
	 \begin{align*}
		g: V &\longrightarrow \mathbb{Z}_{\geq 0} \\
		g(v) &=\text{geometric genus of }X_v  
	\end{align*}
	(I think the geometric genus is the genus of the normalization of the variety.)	
\end{itemize}
This is the combinatorial data attached to the curve.

We 
We look for a general curve with respect to the geometric genus, and say it is \textit{\textbf{general for its position}} if the nodal points are in general position.

Then we have a stratification of 
\[\overline{\mathcal{M}}=\{\text{stable curves } X \text{ with finite automorphism group}  \}\]
And here \textit{stable} means nodal. So for example you can have stability if the degree of $\omega_C|_{X}$ is positive.

So the stratification is:
\[\overline{\mathcal{M}}_{g}=\bigsqcup \mathcal{M}_{(G,g)}\]
where
\[\mathcal{M}_{G,g}=\{X\text{ s.t. $G$ is the dual graph of $X_\omega$ genus function $g$.} \}\]
And we have that
\[ \operatorname{codim}M_{G,g}=|E|,\]
the number of edges.

\begin{remark}
	These are graph curves. All their components are $\mathbb{P}^1$ and they intersect in prescribed way.
\end{remark}

\section{Some combinatorics}

We have the \textit{\textbf{genus formula}}:
 \[P_a(X)=\sum g_v+g(G)\]
\[g(G)=|E|-|V| +1\]
\[g=3g-3-(2g-2)+1\]
\[\operatorname{deg}(F)=4\]

\begin{idea6}{Remark}\leavevmode
When you have maximum number of edges, you force everything to be of a particular kind by the genus formula. (This follows from stability condition.)
\end{idea6}

\begin{remark}
	We may check stability looking if the canonical bundle is trivial.
\end{remark}

\begin{idea1}{Remark}\leavevmode
	Stable $\iff$ every component which is $\mathbb{P}^1$ has at least 3 special points.
\end{idea1}

\section{What we do}

Now let's finish the statement we started before:

Given a nodal curve $X$ which is general for its topology, we describe all limits of the canonical series in any degeneration to $X$, and construct a parameter space for them.

\begin{exercise}
	Let $X$ be a smooth projective variety such that $K_X$ is ample. Prove that its automorphism group is finite.
\end{exercise}



We would like to study the moduli of stable curves. So we have a parameter for the objects we want to classify. Diaz-Cutievman described the locus of curves with special  Weierstrass points.

Weierstrass points are such that the line (what line?) intersect the curve in at least (some bound). So for a quartic,
\[P\text{ is a W point }\iff I(P;T_pC\cap C)\geq 3 \]
and in fact
\[g^3-g=24\text{ W points.}\]

\begin{remark}
	A general smooth curve of genus 3 has exactly 24 Weierstrass points.
\end{remark}

\begin{exercise}\leavevmode 
	The genus $g$ curve has $g^3-g$ Weierstrass points.
\begin{enumerate}[label=\alph*.]
	\item For $g=3$ and plane quartics.
	 \item For hyper-elliptic curve of genus $3$ (also define W point in this case).
\item For non-hyperelliptic genus 4 curve.
\item Etc.
\end{enumerate}
\end{exercise}

\section{The additional data (?)}

Take your variety. The drawing is a bunch of blue lines. Take another line (red). How do the y intersect?
\[\lim_{t\to 0}X_t\cap L=X\cap L.\]
And intersection with another curve $F$?
\[\lim_{t \to 0} X_t\cap L_0=F\cap L_0\]
Let $L=L_0$. We have
\begin{align*}
	L_0L_1L_2L_3+tF&=0\\
	L_0&=0
\end{align*}
Dividing by $t$,
\begin{align*}
	LL_1L_2L_3+F&=0\\
	L_0&=0
\end{align*}
Now look at linear series generated by $LL_1L_2L_3\forall L$ and $F$ on $L_0=0$. $L_0=X$.
\begin{align*}
	(\alpha Y+\beta Z)L_1L_2L_3+\gamma F&=0,\qquad (\alpha,\beta,\gamma )\in\mathbb{P}^2\\
	L_0&=0
\end{align*}

\section{After break}

Now we explain how these systems of divisor appear and how we are going to handle them.

The limit of the $\overset{\vee }{\mathbb{P}}^2$ is some divisors.

We are considering a smoothing
\[\begin{tikzcd}
	\mathfrak{X}\arrow[d]\\
B= \Delta_0= \operatorname{Spec}k[[t]]\subset \mathbb{C}
\end{tikzcd}\]
And we have
\begin{align*}
	\omega_{\mathfrak{X} /B}&\text{ is the relativa canonical bundle}\\
	\omega_{\mathfrak{X} /B}\Big|_{\mathfrak{X}_\eta}=\cap_{\mathfrak{X}_\eta}\\
	\omega_{\mathfrak{X} /B}\Big|_{\mathfrak{X}_\sigma}&=\omega_{X}\subseteq \Omega_X=\bigoplus_{v\in V}\Omega_v  
\end{align*}
where $\Omega_v$ is the space of meromorphic differentials over $C_v=\tilde{X}_v$.

So it's a family of bundles that is the canonical bundle on the general fibers and on the exeptional fiber it is the canonical bundle too.

\section{Regular differentials (Rosenlicht)}

\[(\eta_v)_{v}\in\bigoplus_{v}\Omega_v  \]
\[\operatorname{res}_{p_a}\eta_v+\operatorname{res}_{p_{\bar{a}}}\eta_\omega =0\]
Now $\eta_v$ can only have poles at the branches, and they should be simple.

\begin{remark}
	Looks like we have been computing how the bundles $\mathcal{O}_L(1)$, $\mathcal{O}_{L_i}$ look like when restricted to different subvarieties. So for example
\[\mathcal{O}_{\mathfrak{X}}(1)(-L_0)|_{L_i}=\mathcal{O}_{L_i}(1)(-1)\]
	is just a skyscraper sheaf.
\end{remark}

In the end we concluded that $(L_h,W_h)$ has infinitely many linear series in $X$ where
\[W_h= \Gamma(\mathfrak{X},\mathcal{L}_{h})\Big|_{\mathfrak{X}_0}\cap \Gamma(X,L_h)\]
So, importantly,
\[\{0\neq  s\in W_h\qquad Z(s)\subseteq X\qquad |Z(s)| <\infty\} =\qquad \text{lim divisors} \]
$h$ vary.

\begin{align*}
	\omega_X&\subseteq \Omega =\bigoplus\Omega_v\\
	\omega_{\mathfrak{X}}\left( -\sum h(v)X_{v} \right) \Big|_{X}&\subseteq \Omega \\
	\Omega_{X_{ \omega}}\left( \sum p_u-\sum p_u \right) 
\end{align*}

\begin{idea6}{Canonical case}\leavevmode
	\[W_h\subseteq \Omega =\bigoplus\Omega_v  \]
	where $\Omega_v$ is the space of meromorphic differentials on $C_v=\tilde{X}$. $\dim W_h=g$.
\end{idea6}

\section{Kapranov}
Take some $k$-vector spaces $U_v$ and consider
 \[U=\bigoplus_{v\in V} U_v \]
 and take the grassmanian of subspaces of dimension $g$, and the torus action:
 \[\operatorname{Gr}(g,U)\curvearrowleft  \mathbb{G}^\vee_m=\{\psi:V\to k^*\} \]
 and the projection maps:
 \[\theta_I:\bigoplus_{v\in V} U_v\longrightarrow \bigoplus_{v\in I} U_v,\qquad I\subseteq V  \]
$W$ general, $\theta_I\Big|_{W}$ has maximal rank.

So consider the orbit, it is a Chow variety:
\[\overline{[\mathbb{G}^\vee_m\cdot W}\in\operatorname{Chow}(\operatorname{Gauss}(g,U))\]
And we have the Chow quotient/Hilbert quotient/Mumford quotient (studied first by Thaddeus):
\[\overline{\{\overline{[\mathbb{G}^\vee_m\cdot W]},W\text{ general} \}} \subseteq \operatorname{Chow}(\operatorname{Gauss}(g,V))\]

And then
\[\partial\overset{\{*\}} =\sum [\mathbb{G}^\vee_m\cdot W_i],\qquad \operatorname{Gauss}(g,V)\]
The polytopes asrouled to $W_i$ form a polyhedral decomposition of a certain polytope. Now
\begin{align*}
	W\subseteq U\implies  \mu_W:2^V\longrightarrow \mathbb{Z}
\end{align*}
which is a submodular function,
\[\mu=\mu_W(I)=\dim _k\theta_I(W)\]
\[\mu(I)+\mu(S)\geq \mu(I\cap J)+\mu(J\cup J)\]
\[P_\mu=\{q\in\mathbb{R}^{\vee }:q(I)\leq \mu(I)\;\forall I,\; q(V)=\mu(V)\}\]

\begin{idea5}{What Kapranov observed}\leavevmode
	\[K:\bigcup P_{W_i}=P_{W\text{ general} } \]
\end{idea5}



\end{document}
