\input{/Users/daniel/github/config/preamble.sty}%available at github.com/danimalabares/config
\input{/Users/daniel/github/config/thms-eng.sty}%available at github.com/danimalabares/config

\begin{document}

\begin{minipage}{\textwidth}
	\begin{minipage}{1\textwidth}
		Semin\'ario das Sextas \hfill PUC-Rio
		
		{\small\href{https://github.com/Friday-seminar/}{github.com/Friday-seminar}\hfill\href{https://github.com/danimalabares/seminars}{github.com/danimalabares/seminars}}
		\end{minipage}
\end{minipage}\vspace{.2cm}\hrule

\vspace{10pt}

{\Huge PDE without tears for geometry students}


\hfill{\Large Raffaele Vitolo}

\hfill{\Large Dipartimento di Matematica e Fisica 'E. De Giorgi'}

\hfill{\large November 29 2024}

\section{Part 1}

Conceptualization: S. Lie. Formalization: Eheresmann.

P.D.E. is 
\begin{align*}
x^k \qquad \lambda&=1,\ldots,\text{independent variables} \\
u^i\qquad i&=1,\ldots,m \text{depdendent variables} 
\end{align*}

\begin{align*}
F^1(x^k,u^i,u^i_J)&=0\\
\vdots \\
F^k(x^k,u^i,u^i_J)=0
\end{align*}
where
\[u^i_J=\frac{\partial^{|\sigma|}u^i}{(\partial u^i)^{\sigma_1}\ldots(\partial x^m)^{\sigma_m}}\]
for some multi index $\sigma \in \mathbb{N}^n\sigma=(\sigma_1,\ldots,\sigma_m)$.

In general the $u^i$ are functions; here we may think they are coordinates of some space since we want a geometric introduction to PDEs.

Let $N$ be the spaces of independent variables, $\dim N=n$, and  $M$ the space of dependent variables of dimension $m$. Fields-unknowns are $u:U \subset N \to M$, $u=(u^i)=(u^1,\ldots,u^m)$.

Now we can consider the field as a section of the trivial bundle: $s:N \to N\times M$, $s(x^k)=(x^k,u^i(x^k))$

Now we introduce the notion of contact, which has to do with the derivatives.

\textit{\textbf{1-contact}} of sections $s_1,s_2:N \to N \times M$ at $x_0  \in N$:
\begin{enumerate}
	\item[(0)] $s_1(x_0)=s_2(x_0)$.
	\item (The sections have two graphs that share the tangent plane) $\frac{\partial u^ii_1}{\partial x^k}(x_0)=\frac{\partial u^i_2}{\partial x^k}(x_0)$.
\end{enumerate}

Higher contact:
\[\frac{\partial^{|\sigma|}u^i_1}{(\partial x^1)^{\sigma_1}}\ldots(\partial x^m)^{\sigma_m}(x_0)=\frac{\partial^{|\sigma|}u_2^i}{(\partial x^1)^{\sigma_1}\ldots(\partial x^u)^{\sigma_n}}(x_0)\]
$|\sigma|=\sum_{i=1}^m \sigma_i$, \textit{\textbf{order of derivative}},  $\forall  \sigma \in \mathbb{N}^m$, $0\leq  | \sigma|\leq  r$. This is \textit{\textbf{$r$-th contact}}.

(if the Taylor polynomial unit a certain point is the same)

General notion of contact. $E$ an  ($n+m$)-dimensional manifold and  $L_1,L_2 \subset E$ two $n$-dimensional submanifolds. They have a contact of order $r$ at $x_0$ if there exists a chat $(\underbrace{u^k}_{n},\underbrace{u^i}_{m})$ such that
\begin{itemize}
\item $L_1$ is the graph $(x^1)\to (x^k,u^i_1(x^k))$ and
\item $L_2$ is the graph $(x^1) \to (x^k,u^i_1(x^k))$
\end{itemize}
have a contact of order $r$. (If the manifolds can be seen as graphs in the same chart.)

Now consider the class of all submanifolds touching at the point $x_0$ of order $r$, $[L]_{x_0,r}$. Now, the \textit{\textbf{jet space of order $r$ at $x_0$ of $(E,n)$, $n$ independant variables}} is
\[\bigcup_{x_0 \in E}[L]_{x_0,r}:=J_r(E,n) \]
That's a bundle. The fiber at each point is all the equivalence classes of manifolds that touch at contact etc.
\[\begin{tikzcd}
	\cdots \arrow[r]&J_2(E,u)\arrow[r]& J_1(E,n) \arrow[ r] & E\\
	\cdots \arrow[ r ]&  \left[L\right]_{x_0,2}\arrow[r, maps to] &  \left[L\right]_{x_0,1}\arrow[r,maps to]& x_0
\end{tikzcd}\]

\begin{remark}\leavevmode
	This bundle is in general a Grassmannian.
\end{remark}

Coordinates on $J_2(E,u):$ $(x^1,u^i,J)$
\begin{enumerate}
\item $E \to N$ is a further bundle (e.g.  $N\times M \to N$).
\item $E$ is just a  manifold.
\end{enumerate}

Each has to be dealt with in a way:
\begin{enumerate}
\item For the trivial bundle $N\times M \to N$: Transformations of the tyupe $(u^\mu,v^j)$,
	\[\begin{cases}
		y^\mu=Y^\mu(x^k) \\
		v^j=U^j(x^k)
	\end{cases}\]

For a bundle $E \to N$ 
\[\begin{cases}
		y^\mu=Y^\mu(x^k) \\
		v^j=U^j(x^k, u^i)
	\end{cases}\]

\item 
	\[\begin{cases}
		y^\mu=Y^\mu(x^k,u^i) \\
		v^j=U^j(x^k,u^i)
	\end{cases}\]
\end{enumerate}

\begin{defn}\leavevmode
	A \textit{\textbf{partial differential equation of order $r$}} is a subbundle $\mathcal{E}\subset J_r(E,m)$.
\end{defn}

\begin{example}\leavevmode
$n=m=1$, $u_x=1$. $\mathcal{E}\subset J_1(E,1)$, $E= \mathbb{R} \times \mathbb{R}$, so $J_1(E,1)=\mathbb{R}^3$.

Now take $u(x)=u+c$. We get  $j_1u:N \to J_1(E,1)$, $ u_0 \mapsto [u]_{x_0,1}$.
\end{example}

\begin{example}\leavevmode
	Riemann equation, or Hopf equation, or Inviscid Burgers equation.
	\[u_t+\underbrace{u}_{\text{velocity of wave} } u_x=0\]
	$u$ depends on $t$ and $x$. $n=2$,  $m=1$,  $r=1$.

\begin{quotation}
	…the speed is proportional to the ammount of cars…
\end{quotation}
Let's introduce a variable
\[\xi:=x-tu.\]
We can complement this variable with two other variables that are unchanged:
\[\begin{cases}
	\tilde{t}=t\\
	\xi=x-tu\\
	v=u
\end{cases}\]
now let's take derivatives so that we find the protagonists of Riemann equation:
\[\begin{cases}
	u_x=v_{\tilde{t}}\tilde{t}_x+v_\xi \xi_x=v_\xi(1-tu_x)\\
u_t=v_{\tilde{t}}\tilde{t}_t+v_\xi \xi_t=v_{\tilde{t}}+v_\xi(-u-tu_t)
\end{cases}\]
Now we want to express some in terms of others:
 \[u_x=\frac{v_\xi}{1+\tilde{t}v_\xi},\qquad  u_t=\frac{v_{\tilde{t}}-v v_i}{1+\tilde{t}+v_3}\]
 Now condier
 \[v\frac{v_\xi}{1+\tilde{t}v_\xi}+\frac{(v_{\tilde{t}}-v v_\xi}{1+\tilde{t}v_\xi}=0\]
 But things cancel (which?) and we get
 \[\frac{v_{\tilde{t}}}{1+\tilde{t}v_\xi}=0\]
 so we have a singularity given by
 \[1+\tilde{t}v\xi=0 \iff v_\xi=-\frac{1}{\tilde{t}}\]
 and outside the singularity we have an implicit solution of the Riemann equation:
 \[v=f(\xi),\qquad u=f(x-tu).\]
\end{example}

\begin{upshot}\leavevmode
	Look for symmetries and conservations laws, then a linearizing procedure that allows you to find a solution.
\end{upshot}

\section{Part 2}

Suppose we are given a submanifold $\mathcal{E} \subset J_r(E,u)$ of a jet space. Given $L \subset E$ an $n$-dimensional submanifold, we denote by $j_rL: L \to J_r(E,n)$
\begin{align*}
	j_rL: L &\longrightarrow J_2(E,u) \\
	x_0 &\longmapsto [L]_{x_0,r}
\end{align*}
 \[L:(x^k)\mapsto (x^k,u^i(x^k))\]
 \[j_2L:(x^k)\mapsto (x^k,u^i_\sigma(x^k))\]

 \begin{quotation}
 	$L$ is a soution of $\mathcal{E}$ if and only if $j_rL:L \to \mathcal{E} \subset J_r(E,n)$.
 \end{quotation}
 
\[\mathcal{C}=\{Tj_rL(TL):L \subset E \text{ an $n$-dimensional submanifold} \}\subset T J_r(E,u)\]
So a subbundle of the tangent bundle of the jet-space. Its a distribution of prolonged…

Coordinates: 
\[s(x^k)=(x^k, u^i(x^k))\]
\[T_{j_r}s\left( \frac{\partial }{\partial x^k} \right) =\frac{\partial }{\partial u^k}+\frac{\partial u^i}{\partial u^k}\frac{\partial }{\partial u^i}+\frac{\partial^2 u^i}{\partial x^k \partial x^\mu}\frac{\partial }{\partial u^i_\mu}+\ldots\]

One can prove that $\mathcal{C}$ is generated by:
\[\mathcal{C}=\operatorname{s p a n}\left\{ D_\lambda ,\frac{\partial }{\partial u^i_2}\right\}\]
Where
\[D_\lambda=\frac{\partial }{\partial x^k}+u^i_{\sigma+k}\frac{\partial }{\partial u^i_J},|\sigma|\leq r-1 \]
are called \textit{\textbf{total derivatives}}. They are vector fields in the jet space.
 
Now if $|\tau|=r$,
\[[D_\lambda,D_\mu]=0,\qquad \left[ \frac{\partial }{\partial u^{i_1}_{\tau_1}},\frac{\partial }{\partial u^{i_2}_{\tau_2}} \right] =0\]
and
\[\left[ D_\mu,\frac{\partial }{\partial u^i_x} \right] \neq 0.\]

\section{Part 3}

Now we introduce \textit{\textbf{Lie maps}} which are bundle morphisms  $\phi:J_r(E,n) \to J_r(E,n)$ that preserve the contact distribution of each bundle, i.e.
 \[T\phi(\mathcal{C}_2)\subset \mathcal{C}_r.\]

 \begin{thm}[Lie-Backlund]\leavevmode
 $m>1$. All the morphisms have the form $\phi=\bar{\phi}^{(r)}$, $\bar{\phi} =E \to E$ defined as follows
 \[\bar{\phi}^{(r)}(j_2L)=j_r(\phi \circ L)\]
 \end{thm}

See Kraslkschik-Vigradov: symmetries  and conservation laws of PDE of mathematical physics.

Now
\begin{align*}
\bar{\phi}^{(r)}(j_2L)&=j_r(\phi \circ L)
\end{align*}
For $m=1$, $\phi=\phi_1$. We have a contact transformation
\[\bar{\phi}_1:J_1(E,n) \longrightarrow J_1(E,n),\qquad  \dim E= u+1\]


\begin{quotation}
	Contact transformation help us linearize equations, like the Monge-Ampère equation $u_{xx}u_{y y}-u^2_{xy}=1$, which is a Hessian, using Euler transformation. See Kushner-Lychagin-Roubtsov
\end{quotation}

Now lets define a \textit{\textbf{symmetry}} of  $ \mathcal{E}$. It is a map $\phi:\mathcal{E}\to \mathcal{E}$ such that $T\phi(\mathbb{C}|_{\mathcal{E}})=\mathbb{C}_{\mathcal{E}}$.

Now take a vector field tangent to the $\mathcal{E}$quation: $X: \mathcal{E} \to T\mathcal{E}$ and $L_X\mathcal{C}|_{\mathcal{E}}\subset \mathcal{C}_{\mathcal{E}}$.

How to find symmetries of a PDE? Simplest approach: point symmetries. $X=\bar{X}^{(r)}$.
\[\begin{cases}
	TF(X)=0\\
	\mathcal{E}:F=0
\end{cases}\]

So $X_1,X_2$ symmetries, then $[X_1,X_2]$ is a symmetry.

Why symmetries? for $n=1$ ODE,

 \begin{thm}[Lie-Bianchi]\leavevmode
If there exists a solvable lie algebra o symmetries of an ODE, then the ODE is solvavle by quadratures.
\end{thm}
This is basically reduction of order.

\begin{upshot}\leavevmode
	Finding symmetries simplifies equation, e.g. less variables or turning PDE to ODE.
\end{upshot}

\section{Part 4}

In order to continue with symmetries, we want to look at vector fields on jets and prolonged vector fields of jets. This is a vector field on jets

\begin{align*}X&=X^k \frac{\partial }{\partial x^k}+x^i_\sigma \frac{\partial }{\partial u^i_\sigma}:J_r(E,u)\to T J_r(E,u)
&=x^kD_\lambda+(X^i_\sigma-u^i_{J +\lambda}X^\lambda)\frac{\partial }{\partial u^i_J}
\end{align*}

\begin{equation}\label{eq:1}
X=\bar{X}\iff (X^i_\sigma-u^i_{J+\lambda}X^k)=D_\sigma(X^i-u^i_\lambda X^k)
\end{equation}
\begin{align*}
\bar{X}&=X^\lambda\frac{\partial }{\partial X^\lambda}+X^i\frac{\partial }{\partial u^i}=X^\lambda\left( \frac{\partial }{\partial u^i} +u^i_\lambda\frac{\partial }{\partial u^i}\right) +(X^i-u^i_\lambda X^\lambda)\frac{\partial }{\partial u^i}
\end{align*}
You can split it in a part that uses total derivatives (vertical, I guess) and a part that uses horizontal derivatives. So it's not a actual connection, it's a partial connection. More explicitly,
\[T E \times_E J_1(E,u)=H_1 \oplus_{J_1(E,u)}V_1\]
And if the vector field is a prolongation of a vector field, then \textit{the vertical part must preserve \cref{eq:1}}. 

A \textit{\textbf{generlized symmetry}} is
\[TF(X_v)=\frac{\partial F^k}{\partial u^i_J}\underbrace{D_\sigma(X^i-u^i_\lambda X^\lambda)}_{\ell_F(\bar{X}_V)}=0\]
In other words, a generalized vector field $\varphi:J_R(E,u)\to VE$, $\varphi=\varphi^i \frac{\partial  u^i}{\partial }$.

Now let's do
\begin{equation}\label{eq:2}F=u_t+u u_x + u_{x x u}=0\end{equation}
So
\[\frac{\partial F}{\partial u}=u_x, \frac{\partial F}{\partial u_t}=1,\qquad \frac{\partial F}{\partial u_x}=u	,\qquad \frac{\partial F}{\partial u_{x x x}}=1\]

Let us compute the linearization operator $\ell_F$:
\[\frac{\partial F}{\partial u^i_J}D_J (\varphi)=u_x\varphi+D_t \varphi+u D_x \varphi+ D_{x x x }\varphi=0.\]
And then
 \begin{equation}\label{eq:3}\varphi=\varphi^u(t,x,u)-u_t \varphi^t(t,u,u)-u_x \varphi^x(t,x,u)\end{equation}
 \[\varphi=\varphi(t,x,u,u_x).\]
Now suppose that $\varphi=u_t$. Let's insert it in  \cref{eq:3} and see that goes to zero. Of course, we must the \cref{eq:2}. $\mathsf{OK}$:
\[\ell_F(u_t)=u_xu_t+\ldots \ldots \ldots =0\]
usually you do this with computer.

\section{Conservation laws}

They are differential forms that are closed on the equation.

So for example, for KDV, $\alpha_x$ is \textit{\textbf{density}} and  $\alpha_t$ the \textit{\textbf{flux}}.
 \[\alpha=\alpha_x dx+\alpha_x dt\]
 \[\alpha_x=\alpha_x (t,x,u,u_t,u_x,\ldots)\]
 \[\alpha_t=\alpha_t(t,x,u,u_t,u_x,\ldots)\]
 So we want
 \[d_{t t}\alpha=D_t\alpha_x dt \wedge dx+ D_x \alpha_t dx \wedge dt\Big|_{F=0}=0\]
\[D_t\alpha_x-D_x \alpha_t \Big|_{F=0}=0\]
\[d_H\rho=D_x\rho dx-D_t\rho dt\]
\[d^2_H\rho=0\]
We integrate:
\[D_t \int_{a}^b\alpha_x dx=\int_{a}^b D_t \alpha_x dx=\int_{e}^b D_x \alpha_t dx=\alpha_t|_{a}^b\]



 
\begin{upshot}\leavevmode
	This is the higher-dimensional version of Poisson brackets, first integrals…
\end{upshot} 
 


\end{document}
