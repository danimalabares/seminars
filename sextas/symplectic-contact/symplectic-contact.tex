\input{/Users/daniel/github/config/preamble.sty}%available at github.com/danimalabares/config
\input{/Users/daniel/github/config/thms-eng.sty}%available at github.com/danimalabares/config

\begin{document}

\begin{minipage}{\textwidth}
	\begin{minipage}{1\textwidth}
		Semin\'ario das Sextas \hfill PUC-Rio
		
		{\small\href{https://github.com/Friday-seminar/}{github.com/Friday-seminar}\hfill\href{https://github.com/danimalabares/seminars}{github.com/danimalabares/seminars}}
		\end{minipage}
\end{minipage}\vspace{.2cm}\hrule

\vspace{10pt}

{\Huge Symplectic and contact nature of Riemannian geometry}

\hfill{\Large Graham Smith}

\hfill{\Large PUC-Rio}

\hfill{\large 1st and 8th of November 2024}

Surfaces of constant extrinsic curvature $k$.

\begin{table}[H]
	\centering
	\begin{tabular}{c |c| c| c}
	&$\mathbb{H}^2$ &$\mathbb{R}^3$&$S^3$\\\hline
		$k>1$&  $S^3$&$S^2$&$S^2$\\
		$k=1$&Horospheres, horocycles& &\\
		$0<k<1$&ess. param., $\mathcal{H}(\mathbb{D})\sqcup \mathcal{H}(\mathbb{C})\setminus \{\mathbb{C}\}$
	\end{tabular}
\end{table}

\begin{remark}\leavevmode
	Intrinsic curvature = extrinsic curvature + sectional curvature of the ambient space.
\end{remark}

$X$ a 3-manifold, take $x \in X$ and $v_x \in TX$. The Levi-Civita connection provides
\begin{align*}
T_vTX&\cong H_vTX\oplus V_xTX\\&\cong\underbrace{TxX}_{\text{hor.} }\oplus \underbrace{T_xX}_{\text{vert.} }
\end{align*}
\begin{itemize}
	\item Saski metric \[\left<(\xi,\mu)(\xi',\mu'\right>=\left<\xi,\xi'\right>+\left<\mu,\mu'\right>\]
	\item Symplectic form
		\[\omega\Big((\xi,\mu),(\xi',\mu')\Big)=\left<\xi,\mu'\right>-\left<\xi',\mu\right>\]
		which we may pull back to $X$ via the musical isomorphism to obtain the \textit{\textbf{Saski symplectic form}} of  $T^*X$ $\omega=\flat^*\omega_{\operatorname{st}}$.
\item \[m\Big((\xi,\mu),(\xi',\mu')\Big)\]

	\item There is a complex structure $I$
	
	\item And a quadratic form
	\[\mathcal{m}=\begin{pmatrix}0&I_0\\ I_0&0\end{pmatrix}\]
\end{itemize}
And a contact bundle pulling the luiville form from $T^*X$. And then there is a another complex structure $J$. So define $K:=I J$, which gives a \textbf{hyperkähler contact structure} on the unit tangent bundle. 

\begin{upshot}\leavevmode
	We have created a pseudoholomorphic curve from a $k$-surface by taking the unit tangent normal. That is, consider the normal map as an embedding of our surface in the unit tangent bundle.
\end{upshot}

\begin{thm}[Jurgens]\leavevmode
	$\Sigma\subset\mathbb{R}^4$
	\begin{enumerate}[label=(\roman*)]
	\item complete,
	\item $J$-holomorphic
\item $m|_{T\Sigma}\geq 0$
\end{enumerate}
then $\Sigma$ is a plane. (Morally, the graph of a linear function.)
\end{thm}

\begin{thm}[dim 3,4 Calabi, dim 5$\geq $ Pgorelov]\leavevmode
$f:\mathbb{R}^2\to\mathbb{R}$ 
\begin{enumerate}[label=(\roman*)]
\item $f$ convex,
\item $\det\operatorname{Hess}(f)=1$ (Monge-Ampère)
\end{enumerate}
Then $f$ is quadratic.

\begin{proof}\leavevmode
Short, done in seminar.
\end{proof}
\end{thm}

And then we want to prove a compactness property. So we take a sequence of surfaces $(\Sigma_m,p_n)\subset S^1X$ with $p_n \in \Sigma_n$. Suppose $\|\operatorname{I I}_m\|\xrightarrow{m \to \infty}\infty$. Then there exists $B_m$ and  $q_m$ such that
 \begin{enumerate}[label=(\roman*)]
\item $\| \operatorname{ I I}_m(q_m)\|=B_m$.
\item $B_m\xrightarrow{m \to \infty}\infty$ and,
\item by a lemma that is easy to prove, $\forall  r\in B_{\frac{1}{2\sqrt{\|\operatorname{I I}_m(q_m)\|} }}(q_m)$.
\end{enumerate}
And then we rescale the metrics $g \to g_m=B^2_mg$. That makes the metric be flatter and flatter, like zooming in, and also makes the shape operator of every surface have norm 1. and we get
\[(S^1X,g_m)\longrightarrow(\mathbb{R}^5,)\]
\[\Sigma_m\longrightarrow \Sigma_m \subset\mathbb{R}^5\]
with respect to the famous Cheeger Gromov topology in the limit, which is not so easily defined. And by Arzelá-Ascoli and Elliptic regularity magic (see M. Joshi Course notes) (regularity is not smoothness but some differentiability) the limit surface is compact, positive, $J$-homolomorphic. And by Jurgen's theorem they are flat. But that's a contradiction with the fact that the shape operators of these surfaces have norm 1.

So you have compactness. Yaaaaay!

\end{document}
