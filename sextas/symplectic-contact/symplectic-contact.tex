\input{/Users/daniel/github/config/preamble.sty}%available at github.com/danimalabares/config
\input{/Users/daniel/github/config/thms-eng.sty}%available at github.com/danimalabares/config

\begin{document}

\begin{minipage}{\textwidth}
	\begin{minipage}{1\textwidth}
		Semin\'ario das Sextas \hfill PUC-Rio
		
		{\small\href{https://github.com/Friday-seminar/}{github.com/Friday-seminar}\hfill\href{https://github.com/danimalabares/seminars}{github.com/danimalabares/seminars}}
		\end{minipage}
\end{minipage}\vspace{.2cm}\hrule

\vspace{10pt}

{\Huge Symplectic and contact nature of Riemannian geometry}

\hfill{\Large Graham Smith}

\hfill{\Large PUC-Rio}

\hfill{\large 1st and 8th of November 2024}

Surfaces of constant extrinsic curvature $k$.

\begin{table}[H]
	\centering
	\begin{tabular}{c |c| c| c}
	&$\mathbb{H}^2$ &$\mathbb{R}^3$&$S^3$\\\hline
		$k>1$&  $S^3$&$S^2$&$S^2$\\
		$k=1$&Horospheres, horocycles& &\\
		$0<k<1$&ess. param., $\mathcal{H}(\mathbb{D})\sqcup \mathcal{H}(\mathbb{C})\setminus \{\mathbb{C}\}$
	\end{tabular}
\end{table}

\begin{remark}\leavevmode
	Intrinsic curvature = extrinsic curvature + sectional curvature of the ambient space.
\end{remark}

$X$ a 3-manifold, take $x \in X$ and $v_x \in TX$. The Levi-Civita connection provides
\begin{align*}
T_vTX&\cong H_vTX\oplus V_xTX\\&\cong\underbrace{TxX}_{\text{hor.} }\oplus \underbrace{T_xX}_{\text{vert.} }
\end{align*}
\begin{itemize}
	\item Saski metric \[\left<(\xi,\mu)(\xi',\mu'\right>=\left<\xi,\xi'\right>+\left<\mu,\mu'\right>\]
	\item Symplectic form
		\[\omega\Big((\xi,\mu),(\xi',\mu')\Big)=\left<\xi,\mu'\right>-\left<\xi',\mu\right>\]
		which we may pull back to $X$ via the musical isomorphism to obtain the \textit{\textbf{Saski symplectic form}} of  $T^*X$ $\omega=\flat^*\omega_{\operatorname{st}}$.
\item \[m\Big((\xi,\mu),(\xi',\mu')\Big)\]

	\item There is a complex structure $I$
	
	\item And a quadratic form
	\[\mathcal{m}=\begin{pmatrix}0&I_0\\ I_0&0\end{pmatrix}\]
\end{itemize}
And a contact bundle pulling the luiville form from $T^*X$. And then there is a another complex structure $J$. So define $K:=I J$, which gives a \textbf{hyperkähler contact structure} on the unit tangent bundle. 

\begin{upshot}\leavevmode
	We have created a pseudoholomorphic curve from a $k$-surface by taking the unit tangent normal. That is, consider the normal map as an embedding of our surface in the unit tangent bundle.
\end{upshot}

\begin{thm}[Jurgens]\leavevmode
	$\Sigma\subset\mathbb{R}^4$
	\begin{enumerate}[label=(\roman*)]
	\item complete,
	\item $J$-holomorphic
\item $m|_{T\Sigma}\geq 0$
\end{enumerate}
then $\Sigma$ is a plane. (Morally, the graph of a linear function.)
\end{thm}

\begin{thm}[dim 3,4 Calabi, dim 5$\geq $ Pgorelov]\leavevmode
$f:\mathbb{R}^2\to\mathbb{R}$ 
\begin{enumerate}[label=(\roman*)]
\item $f$ convex,
\item $\det\operatorname{Hess}(f)=1$ (Monge-Ampère)
\end{enumerate}
Then $f$ is quadratic.

\begin{proof}\leavevmode
Short, done in seminar.
\end{proof}
\end{thm}

And then we want to prove a compactness property. So we take a sequence of surfaces $(\Sigma_m,p_n)\subset S^1X$ with $p_n \in \Sigma_n$. Suppose $\|\operatorname{I I}_m\|\xrightarrow{m \to \infty}\infty$. Then there exists $B_m$ and  $q_m$ such that
 \begin{enumerate}[label=(\roman*)]
\item $\| \operatorname{ I I}_m(q_m)\|=B_m$.
\item $B_m\xrightarrow{m \to \infty}\infty$ and,
\item by a lemma that is easy to prove, $\forall  r\in B_{\frac{1}{2\sqrt{\|\operatorname{I I}_m(q_m)\|} }}(q_m)$.
\end{enumerate}
And then we rescale the metrics $g \to g_m=B^2_mg$. That makes the metric be flatter and flatter, like zooming in, and also makes the shape operator of every surface have norm 1. and we get
\[(S^1X,g_m)\longrightarrow(\mathbb{R}^5,)\]
\[\Sigma_m\longrightarrow \Sigma_m \subset\mathbb{R}^5\]
with respect to the famous Cheeger Gromov topology in the limit, which is not so easily defined. And by Arzelá-Ascoli and Elliptic regularity magic (see M. Joshi Course notes) (regularity is not smoothness but some differentiability) the limit surface is compact, positive, $J$-homolomorphic. And by Jurgen's theorem they are flat. But that's a contradiction with the fact that the shape operators of these surfaces have norm 1.

So you have compactness. Yaaaaay!

{\Huge Magnetic monopoles according to Hitchin}

\hfill{\Large Graham Smith}

\hfill{\Large PUC-Rio}

\hfill{\large 8 of February 2024}

\begin{defn}\leavevmode
A \textit{\textbf{magnetic monopole}} consists of 3 elements: \((E,\nabla,\Phi)\), a bundle, a connection, and the Higgs field.
\end{defn}

And the Bogomolny equations are:

\begin{lemma}[Bogomolny equations]\leavevmode
\(\nabla\) is anti self dual iff \(* F^{\nabla^1}=\nabla^1\Phi_0\).
\end{lemma}

\begin{thing7}{What's going on}[I arrived in late]\leavevmode
ChatGPT says \(F^\nabla\) is the curvature of this connection on this principal bundle. So Bogomolny equations are equations on the curvature. Like what? \textbf{Like Maxwell's equations!} 
\end{thing7}

Here's some more ChatGPT on what's about to come in this talk:

\textbf{Magnetic Monopoles and Twistors.} A magnetic monopole consists of a vector bundle \(E\), a connection \(\nabla\), and a Higgs field \(\Phi\). The Bogomolny equation, \(*F^\nabla = \nabla \Phi\), describes monopole solutions balancing curvature and the Higgs field. The Higgs field \(\Phi\) can be interpreted as a difference between two connections or as a section of \(\text{End}(E)\).

Twistor theory provides a complex-geometric approach to gauge fields. The twistor space of \(\mathbb{R}^3\) can be described in terms of oriented lines, isotropic \(\mathbb{C}\)-lines, or null \(\mathbb{C}\)-planes in \(\mathbb{C}^3\). For a \(\mathsf{U}(2)\)-connection, one constructs a holomorphic vector bundle \(\tilde{E} \to X\) over the complex manifold \(X = TS^2\), encoding monopoles in terms of holomorphic structures.


\begin{remark}[Misha]\leavevmode
\(\Phi\) is sections of endomorphisms of the group which is \(G\).
\end{remark}

\begin{question}\leavevmode
What is the Higgs field? It is a \textit{difference} between two connections.
\end{question}

\begin{question}\leavevmode
What are twistors?
\end{question}

\subsection{Twistors in \( \mathbb{R}^3\)}

Embed \(\mathbb{R}^3 \hookrightarrow  \mathbb{C}^3\). Extender the inner product of \(\mathbb{R}^3\) \(\left<\cdot ,\cdot \right>\) \textit{blinearly} to \(\mathbb{C}^3\). (It's not hermitian, it's a bilinear form.)

Twistor space of \(\mathbb{R}^3\):

\begin{enumerate}
	\item Oriented lines in \(\mathbb{R}^3\).
\item Null \(\mathbb{C}\)-lines in \(\mathbb{C}^3\) (isotropic).
\item Null \(\mathbb{C}\)-planes in \(\mathbb{C}^3\).
\end{enumerate}
The three are equivalent (we have shown that in the seminar).

It turns out that this space can be parametrized by \(TS^3\) as follows:
\[L^+ \to \underbrace{x}_{\substack{\text{unit}  \\ \text{tangent} } }\qquad \underbrace{y}_{\substack{\text{closest}  \\ \text{point} }}\]
\[x \in S^2\qquad y \perp x \iff y \in T_x S^2.\]
Now we have a complex integrable structre becuase \(S^2=\hat{\mathbb{C}}\)

Now

suppose \((E^2, \nabla, \Phi)\). Fix the group to be  \(\mathsf{U}(2)\) (it's a \(\mathsf{U}(2)\)-connection. Define a bundle \(\tilde{E}\) over \(X\) by
\begin{align*}
\tilde{E}_x&= \{ \sigma: L_x \to R: \nabla_T\sigma - u\Phi \sigma=0\}\\
&=\{\sigma:L_x \to E: \tilde{\nabla}_{(T+i e_0)}\sigma=0\}
\end{align*}
where \(\tilde{\nabla}\) is the Yang-Mills connection.

Now take the complex manifold \(X= TS^2\), so \((E^2,\nabla,\Phi\) and produce the bundle \(\tilde{ E} \to X\). Then you may actually put a holomorphic structure on \(\tilde{E}\), making into a holomorphic vector bundle over \(X\).

Now we write:
\begin{align*}
X&=\{\text{oriented lines in \(\mathbb{R}^3\)} \}\\
T_{(x,y)}TS^2&=\{(u,v): \left< u,v\right>=0, \left<u,y\right>+\left<x,v\right>=0\}\\
V_x&=\{(0,v) : \left<x,v\right>=0\}\\
J(0,v)&=(0,x \times v)\\
H_{(x,y)}TS^2&=\{(u,-\left<u,y\right>x:\left<u,x\right>=0\}\\
\hat{u}:=(u,-\left<u,y\right>x)\\
J\hat{u}&=\widehat{x \times u}
\end{align*}

And then: Jac fields orthogonal to \(L^+\) is 4 dimensional.
\[J_{\xi}=T \times \xi\]
Constant curvature \(\to \) integrable.

\begin{align*}
\gamma_{x,y}(t)&=y+t x\\
(u,v-&\left<u,y\right>x\\
\xi(t)&=v-\left<u,y\right>x+tu
\end{align*}
\(G\) lie group, \(H\subseteq G\)
\begin{align*}
\mathfrak{g}&=\mathfrak{h} \oplus  \mathfrak{p}\\
[\mathfrak{g},\mathfrak{p}] \subseteq \mathfrak{p}\quad [\mathfrak{p},\mathfrak{p}] \subseteq \mathfrak{g}\qquad \text{Polarization}
\end{align*}
\(\mathfrak{p}\) id. with \(T(G/H)\) fwy \(\operatorname{Ad}_\mathfrak{g}\)-invariant form on \(\mathfrak{p}\) generates a parallel form on \(G/H\).

 \begin{align*}
G&=\mathsf{SO}(3)\ltimes \mathbb{R}^3\\
H&=\mathsf{SO}(2) \times \mathbb{R}\\
\mathfrak{p}=4, J, \nabla J=0
\end{align*}

Here's ChatGPT:

\textbf{Twistor Space and Structure.} The twistor space of \(\mathbb{R}^3\) is given by \(X = TS^2\), which parametrizes oriented lines in \(\mathbb{R}^3\). The tangent space at \((x,y) \in TS^2\) decomposes as
\[
T_{(x,y)}TS^2 = \{(u,v) : \langle u,v \rangle = 0, \langle u,y \rangle + \langle x,v \rangle = 0 \}.
\]
The vertical and horizontal subspaces are defined as
\[
V_x = \{(0,v) : \langle x,v \rangle = 0\}, \quad H_{(x,y)}TS^2 = \{(u,-\langle u,y \rangle x) : \langle u,x \rangle = 0\}.
\]
An almost complex structure \(J\) is introduced by
\[
J(0,v) = (0, x \times v), \quad J\hat{u} = \widehat{x \times u}.
\]

\textbf{Jacobi Fields and Integrability.} The Jacobi fields orthogonal to an oriented line \(L^+\) form a 4-dimensional space, where
\[
J_{\xi} = T \times \xi.
\]
Constant curvature ensures integrability.

The geodesic equation for a curve \(\gamma_{x,y}(t) = y + tx\) leads to
\[
\xi(t) = v - \langle u,y \rangle x + tu.
\]

\textbf{Lie Groups and Homogeneous Structure.} Consider a Lie group \(G\) with a subgroup \(H \subseteq G\), and the Lie algebra decomposition
\[
\mathfrak{g} = \mathfrak{h} \oplus \mathfrak{p}, \quad [\mathfrak{g},\mathfrak{p}] \subseteq \mathfrak{p}, \quad [\mathfrak{p},\mathfrak{p}] \subseteq \mathfrak{g}.
\]
Since \(\mathfrak{p}\) is identified with \(T(G/H)\), an \(\operatorname{Ad}_{\mathfrak{g}}\)-invariant form on \(\mathfrak{p}\) induces a parallel form on \(G/H\).

For the twistor space,
\[
G = \mathsf{SO}(3) \ltimes \mathbb{R}^3, \quad H = \mathsf{SO}(2) \times \mathbb{R}.
\]
The space \(\mathfrak{p}\) has dimension 4, and the almost complex structure \(J\) satisfies \(\nabla J = 0\), ensuring integrability.


\subsection{Scattering transform}

Suppose you have a Lie group \(G\), and two Lie subgroups \(H_1, H_2 \subseteq G\). Now take
 \[X:=G/H_1,\qquad X_2=G/H_2\]
\[\begin{tikzcd}
&G\arrow[dl,"\pi_1",swap]\arrow[dr,"\pi_2"]\\
X_1&&X_2
\end{tikzcd}\]

\textbf{Radon transform}:
\[R: C^\infty(X_1) \longrightarrow C^\infty(X_2)\]
So
\begin{align*}
R&=\pi_{2 *}\pi_1^*\\
G&=\mathsf{SO}(3)\ltimes \mathbb{R}^3\qquad  \text{isometries of \(\mathbb{R}^3\)} \\
H_1&=\operatorname{S t a b} \text{ point} = \mathsf{SO}(3)\\
H_2&=\text{Stable line} =S^1 \times \mathbb{R}
\end{align*}
Now we can put a little zero on the continuous functions (that means they tend to zero?):
\[R: C_0^\infty(X_1) \longrightarrow C^\infty(X_2)\]
\begin{align*}
R&=\pi_{2*}\pi_1^* \\
\pi_1^* (f)&=f \circ \pi_1\\
\pi_{2*}f&=\int_HfdV\frac{1}{2\pi}\\
\hat{f}(L)&=\int_Lfde\\
F&=\pi_{2*}\mu_{\phi}\pi_1^*
\end{align*}


\[\begin{tikzcd}
&G=\mathsf{SO}(3)\ltimes \mathbb{R}^3\arrow[dl,"\pi_1",swap]\arrow[dr,"\pi_2"]\\
\mathbb{R}^3&&X\\
&(\pi_1 ^*E,\pi_1^*\nabla,\pi^*_1\phi)\arrow[uu]
\end{tikzcd}\]

So a global section
\[(\pi_1^*E_1|_{F},\pi_1^*\nabla|_{F},\pi_1^*\phi|_{F}\]

\textbf{Summary of the Scattering Transform.} Given a Lie group \( G \) and two subgroups \( H_1, H_2 \subseteq G \), we define the homogeneous spaces
\[
X_1 = G/H_1, \quad X_2 = G/H_2.
\]
The \textit{Radon transform} is a mapping
\[
R: C^\infty_0(X_1) \to C^\infty(X_2),
\]
which factors through the pullback \(\pi_1^*\) and pushforward \(\pi_{2*}\):
\[
R = \pi_{2*} \pi_1^*.
\]

For \( G = \mathsf{SO}(3) \ltimes \mathbb{R}^3 \) (the isometry group of \( \mathbb{R}^3 \)), the subgroups correspond to:
\[
H_1 = \mathsf{SO}(3) \quad \text{(stabilizing a point)}, \quad H_2 = S^1 \times \mathbb{R} \quad \text{(stabilizing a line)}.
\]
The transform integrates functions along fibers:
\[
\pi_1^* f = f \circ \pi_1, \quad \pi_{2*} f = \int_H f dV.
\]
A section of a lifted bundle \((\pi_1^*E, \pi_1^*\nabla, \pi_1^*\phi)\) over \( G \) descends through \( \pi_2 \), encoding the scattering transform in terms of differential geometric data.


\end{document}
